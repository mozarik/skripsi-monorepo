\renewcommand{\chaptername}{BAB}
%-----------------------------------------------------------------------------%
\chapter{METODOLOGI PENELITIAN}
%-----------------------------------------------------------------------------%

\vspace{4.5pt}
\setlength{\parskip}{0.5em}
\section{Rancangan Penelitian}\label{sec:rancangan_penelitian}
Tipe Penelitian yang peneliti gunakan disini merupakan penelitian terapan langsung. Peneliti melakukan
\textit{development} secara langsung terhadap Argo CD yang akan digunakan pada sistem bare-metal (\textit{non-cloud}) yang mengembangkan fitur continous integration dengan ArgoCD

\begin{figure}[h]
    \centering
    \includegraphics[width=1\textwidth]{figures/Tahapan Skripsi.png}
    \caption{Rancangan Penelitian}
\end{figure}

\section{Studi Literatur}\label{sec:studi_literatur}
Pada tahap studi literatur, peneliti mengumpulkan teori-teori,
konsep, dan temuan penelitian terdahulu yang relevan sebagai landasan dalam melakukan pengembangan dan implementasi automasi deployment menggunakan Argo CD pada lingkungan Kubernetes.
Studi literatur ini dilakukan dengan menelaah berbagai sumber seperti buku, jurnal nasional dan internasional, serta dokumentasi resmi terkait GitOps dan Argo CD.

Peneliti mempelajari konsep dasar GitOps yang menekankan penggunaan repository Git sebagai sumber kebenaran (single source of truth) untuk seluruh konfigurasi dan deployment aplikasi \cite{Weaveworks2017}.
Selain itu, peneliti juga mengkaji dokumentasi resmi Argo CD yang menjelaskan fitur, arsitektur, serta best practice dalam penerapannya pada workflow CI/CD \cite{ArgoCDDocs}.
Penelitian terdahulu dari Korhonen \cite{Korhonen2021} menjadi salah satu referensi utama, di mana dijelaskan penerapan Argo CD untuk meningkatkan konsistensi, efisiensi, dan auditability proses deployment aplikasi pada Kubernetes.

Selain itu, peneliti juga menelaah studi komparatif antara Argo CD dan tools GitOps lain seperti Flux yang dilakukan oleh Sharma et al. \cite{Sharma2022},
serta kajian terkait tantangan keamanan dalam implementasi Argo CD yang dibahas oleh Kumar \cite{Kumar2023}. Dengan mengkaji berbagai referensi tersebut,
peneliti memperoleh pemahaman yang komprehensif mengenai teori dan praktik automasi deployment berbasis GitOps, sehingga dapat merancang dan mengimplementasikan solusi yang sesuai dengan kebutuhan penelitian.

\section{Perancangan (Design)}
Setelah melakukan tahap studi literatur dan analisis kebutuhan sistem, tahap selanjutnya adalah melakukan perancangan sistem automasi deployment yang akan diimplementasikan. Pada tahap studi literatur yang telah dilakukan, peneliti mengambil rancangan dasar dari penelitian-penelitian terdahulu, khususnya model GitOps yang dibahas oleh Ramadoni \cite{Ramadoni2021} dan arsitektur Argo CD yang dijelaskan oleh Korhonen \cite{Korhonen2021}.

Pada tahap ini, peneliti melakukan modifikasi terhadap rancangan yang sudah ada untuk mengadaptasi implementasi Argo CD pada lingkungan bare-metal (non-cloud) dengan memanfaatkan platform virtualisasi Proxmox. Modifikasi ini penting dilakukan karena sebagian besar implementasi GitOps dan Argo CD yang ada di literatur berfokus pada lingkungan cloud \cite{Bolscher2019}, sementara penelitian ini bertujuan untuk mengimplementasikan solusi serupa pada infrastruktur on-premise.
Perancangan yang dilakukan meliputi beberapa aspek utama, yaitu:

\begin{enumerate}
    \item Perancangan arsitektur sistem automasi deployment menggunakan Argo CD pada lingkungan Kubernetes yang berjalan di atas Proxmox
    \item Perancangan alur kerja sistem sesuai dengan pendekatan pull-based deployment yang merupakan karakteristik utama dari GitOps \cite{Weaveworks2017}
    \item Perancangan mekanisme continuous integration yang terintegrasi dengan Argo CD untuk membentuk pipeline CI/CD yang lengkap
    \item Perancangan strategi deployment dan rollback yang efektif untuk aplikasi berbasis microservice
\end{enumerate}

\section{Implementasi Code Automasi (Coding)}
Pada tahap implementasi ini dilakukan penulisan kode terhadap beberapa microservice yang akan menjadi usecase yang akan digunakan menggunakan bahasa Go.
Pada tahap ini juga akan dilakukan automasi script untuk membuat infrastruktur dimana ArgoCD akan berjalan. Tahap implementasi akan menghasilkan sebuah sistem automasi, microservices,
beserta infrastructur tempat berjalannya ArgoCD yaitu infrastructure kubernetes.

\section{Pengujian dan Analisis (Testing dan Analysis)}
Dalam pengujian dan analisis hasil dari implementasi yang dibuat akan dilakukan tahap pengujian terlebih dahulu.
Pengujian dan analisis dilakukan untuk mencari dan mengidentifikasi kesalahan pada sebuah sistem. Pengujian akan dilakukan menggunakan black-box testing dengan metode validasi User Acceptance Testing
dengan kriteria usability testing.

\section{Kesimpulan}
Kesimpulan adalah tahapan terakhir setelah dilakukan implementasi dan pengujian. Pada tahap ini peneliti akan membuat kesimpulan untuk bertujuan
mengetahui letak kelebihan dan kekurangan sistem yang telah dikembangkan.
\newpage
