%-----------------------------------------------------------------------------%
\chapter{TINJAUAN PUSTAKA}
%-----------------------------------------------------------------------------%

\vspace{4.5pt}
\setlength{\parskip}{0.5em}

\section{Arsitektur \textit{Microservice}} \label{sec:Arsitektur Microservice}
Arsitektur \textit{Microservice} merupakan sebuah pendekatan atau pola desain dalam membangun sebuah perangkat lunak
yang terbagi-bagi menjadi beberapa komponen perangkat lunak. \textit{Microservice} sendiri merupakan
servis kecil yang bekerja secara independen yang saling berinteraksi dengan service lainnya via \textit{messages protocol} \cite{Newman2015}

\textbf{insert gambar arsitektur microservice}

Sebelum adanya microservice, perangkat lunak dibangun dengan cara \textit{Monolithic}. Secara ringkas, \textit{Monolithic Software} adalah suatu
sistem yang dikembangkan dengan satu kesatuan yang besar dan tidak dapat dieksekusi secara independen \cite{Nicola2017}.
\textit{Monolithic Software} memiliki beragam komponen terikat yang saling berinteraksi dan memiliki
tanggung jawab terhadap alur kerja dari perangkat lunak tersebut.

Terdapat beberapa kekurangan dari pendekatan tersebut dari sisi pengembangan perangkat lunak dan \textit{maintainability}.
Menurut Nicola dkk \cite{Nicola2017}, beberapa kekurangan dari pendekatan \textit{Monolithic Software} adalah:
\begin{itemize}
    \item \textit{Monolithic Software} memiliki kompleksitas yang tinggi, sehingga membutuhkan waktu yang lama untuk
          memahami dan memodifikasi kode programnya.
    \item \textit{Monolithic Software} biasa terdapat \textit{``dependency hell''} yang menyebabkan akan sangat sulit untuk
          menambahkan atau meng-\textit{update library} dan akan berdampak terhadap inkonsistensi sistem.
    \item Perubahan terhadap satu modul dalam sistem monolith membutuhkan sistem tersebut untuk dilakukan \textit{reboot}.
          Hal ini berarti akan mengakibatkan \textit{downtime} pada sistem ketika akan menjalankan sistem dengan versi baru.
    \item \textit{Monolithic Software} terdapat limitasi untuk melakukan scaling.
    \item Terdapat \textit{lock-in} pada teknologi yang akan digunakan.
\end{itemize}

Untuk mengatasi beberapa masalah terkait \textit{Monolithic Software}, diperlukan pendekatan
baru dalam membangun perangkat lunak. Salah satu pendekatan yang dapat digunakan adalah dengan
menggunakan arsitektur \textit{Microservice}.
Prinsip dasar untuk mendesain sebuah perangkat lunak tetap diterapkan pada proses pembangunan microservice itu sendiri.
Perangkat lunak akan tetap dibagi menjadi beberapa komponen. Hal tersebut dilakukan untuk mempermudah dalam
pengembangan \textit{microservice} yang bersifat independen.

\vspace{0.5cm}
\section{Kubernetes} \label{sec:Kubernetes}
Materi pada bab 2.2 ini bersumber dari dokumentasi Kubernetes \cite{kubernetes2021What} dan buku Kubernetes in Action \cite{KubernetesBook}.
Kubernetes adalah \textit{platform} yang berfungis untuk mengatur \textit{deployment} sebuah aplikasi yang sudah
dibuat menjadi sebuah \textit{container} atau dapat disebut dengan \textit{containerized application}.
Kubernetes merupakan sebuah \textit{open-source} yang dikembangkan oleh Google untuk membantu proses orkestrasi pada skala besar.
Pada tahun 2014, Google memutuskan untuk membuat \textit{kubernetes} menjadi proyek \textit{open-sourced} yang
akan dikelola oleh \textit{Cloud Native Computing Foundation} (CNCF).

Pada skala \textit{production} dalam industri, \textit{kubernetes} umumnya digunakan
untuk memudahkan dalam melakukan \textit{deployment} aplikasi agar selalu berada pada kondisi ideal
yang didefinisikan oleh pengguna. Beberapa fungsi \textit{kubernetes} antara lain:
\begin{itemize}
    \item \textit{Deployment} dan \textit{scaling} aplikasi.
    \item \textit{Rolling updates} dan \textit{rollbacks} aplikasi.
    \item \textit{Service discovery} dan \textit{load balancing}.
    \item \textit{Storage orchestration}.
    \item \textit{Automated rollouts and rollbacks}.
    \item \textit{Self-healing} terhadap aplikasi yang berjalan.
    \item Pengaturan konfigurasi dan \textit{secret} yang digunakan oleh aplikasi.
    \item Abstraksi terhadap infrastruktur yang digunakan.
\end{itemize}

Dengan digunakannya \textit{kubernetes}, proses konfigurasi manual dalam \textit{deployment} menjadi tidak diperlukan.
Ini membuat kesalahan dalam proses \textit{deployment} menjadi berkurang sehingga aplikasi menjadi dapat lebih diandalkan dan
seorang \textit{software engineer} mempunyai lebih banyak waktu untuk fokus pada pengembangan aplikasi.

\textbf{insert gambar komponen kubernetes disini}

\textit{Kubernetes} terdiri dari beberapa komponen yang masing masing memiliki fungsi yang berbeda.
Komponen ini saling berkoordinasi untuk melakukan orkestrasi pada \textit{container} yang berjalan.
\textit{Kubernetes} terdiri dari 2 jenis server atau bisa disebut dengan \textit{node}:
\begin{itemize}
    \item \textit{Master Node}
          \textit{Master Node} adalah node yang menjalan komponen yang terdapat pada \textit{control plane}
          yang terdiri dari \textit{etcd, kube-apiserver, kube-scheduler, kube-controller-manager, cloud-controller-manager}.
          yang akan mengkoordinasikan \textit{worker node}
    \item \textit{Worker Node}
          \textit{Worker Node} adalah node yang bertugas untuk mengeksekusi aplikasi yang dijalankan pada \textit{kubernetes}.
          \textit{Worker Node} terdiri dari \textit{kubelet, kube-proxy, container runtime}.
\end{itemize}
Dalam sebuah \textit{cluster kubernetes}, umumnya terdapat 1 master dan banyak node.
Gambar \textbf{nomor gambar} memberikan ilustrasi dari komponen yang terdapat pada \textit{kubernetes}.

%%%%%%%%%%%%%%%%%%%%%%%%%%%%%%%%%%%%%%%%%%%%%%%%%%%%%%%%%%%%%%%%%%%%%%%%%%%%%%%%%%%%%%%%%%%%
% Komponen master
%%%%%%%%%%%%%%%%%%%%%%%%%%%%%%%%%%%%%%%%%%%%%%%%%%%%%%%%%%%%%%%%%%%%%%%%%%%%%%%%%%%%%%%%%%%%
\vspace{0.5cm}
\subsection{Komponen Master Node} \label{subsec:Komponen Master Node}
\textit{Master Node} bertugas untuk menjalankan \textit{control-plane} pada sebuah \textit{cluster kubernetes}.
\textit{Control-plane} sendiri terdiri dari beberapa komponen
\begin{itemize}
    \item \textit{etcd} \\
          \textit{etcd} merupakan \textit{key-value store} yang digunakan untuk menyimpan konfigurasi
          dari \textit{cluster kubernetes} yang terdiri dari informasi tentang \textit{node}, \textit{pod}, dan \textit{service}.
          \textit{etcd} juga digunakan untuk menyimpan informasi tentang \textit{deployment} yang dilakukan pada \textit{cluster kubernetes}.
    \item \textit{kube-apiserver} \\
          \textit{kube-apiserver} merupakan \textit{REST API} yang digunakan untuk berkomunikasi dengan \textit{etcd}
          dan juga dengan \textit{kubelet} yang berjalan pada \textit{worker node}.
          \textit{kube-apiserver} juga digunakan untuk melakukan validasi terhadap \textit{request} yang masuk.
    \item \textit{kube-scheduler} \\
          \textit{kube-scheduler} bertugas untuk melakukan \textit{scheduling} terhadap \textit{pod} yang baru dibuat.
          \textit{Pod} yang baru dibuat akan dijadwalkan pada \textit{node} yang tersedia.
    \item \textit{kube-controller-manager} \\
          \textit{kube-controller-manager} merupakan \textit{daemon} yang berfungsi untuk menjalankan \textit{controller}
          yang digunakan untuk melakukan \textit{management} terhadap \textit{cluster kubernetes}.
          \textit{Controller} yang digunakan antara lain:
          \begin{itemize}
              \item \textit{Node Controller} \\
                    \textit{Node Controller} bertugas untuk melakukan \textit{monitoring} terhadap \textit{node} yang berjalan pada \textit{cluster kubernetes}.
                    \textit{Node Controller} akan melakukan \textit{monitoring} terhadap \textit{node} yang berjalan pada \textit{cluster kubernetes}
                    untuk memastikan bahwa \textit{node} tersebut masih dalam kondisi normal.
                    Jika \textit{node} tersebut mengalami masalah maka \textit{node controller} akan melakukan \textit{eviction} terhadap \textit{pod} yang berjalan pada \textit{node} tersebut.
              \item \textit{Replication Controller} \\
                    \textit{Replication Controller} bertugas untuk mengatur jumlah \textit{pod} pada \textit{cluster}
                    agar sesuai dengan \textit{desired state} yang telah didefinisikan,
              \item \textit{Endpoint Controller} \\
                    \textit{Endpoint Controller} bertugas untuk mengatur \textit{endpoint} yang digunakan untuk mengakses \textit{service}.
                    Dengan ini, ketika jumlah \textit{pod} atau configurasu \textit{pod} yang berjalan pada \textit{cluster kubernetes} berubah,
                    akses kepada \textit{pods} tidak akan berubah karena sudah dilakukan abstraksi oleh entitas \textit{servcie}
              \item \textit{Service Account \& Token Controllers} \\
                    \textit{Service Account \& Token Controllers} bertugas untuk membuat \textit{account} dan \textit{token} yang digunakan untuk mengakses \textit{API server}.
                    Hal ini mengatur akses terhadap \textit{resource} yang ada pada \textit{cluster kubernetes}.
          \end{itemize}
    \item \textit{cloud-controller-manager}
          Komponen ini berguas untuk integrasi fungsi-fungsi spesifik yang disediakan oleh \textit{cloud provider}
          tertentu. Komponen ini memugkinkan fungsi fungsi tambahan dari \textit{cloud provider} yang digunakan
          dapat langsung dijalankan melalui API yang disediakan oleh \textit{cloud provider} tersebut.
          Komponen dari \textit{cloud-controller-manager} antara lain:
          \begin{itemize}
              \item \textit{Node Controller} \\
                    \textit{Node controller} berfungsi memeriksa apakah sebuah \textit{node} telah dihapus ketika \textit{node} tersebut tidak memberi respons.
              \item \textit{Route Controller} \\
                    \textit{Route controller} berfungsi untuk mengatur \textit{route network} pada \textit{cloud provider} yang digunakan.
                    Umumya \textit{cloud provider} akan memiliki konfigurasi \textit{network} yang berbeda-beda.
              \item \textit{Service Controller} \\
                    \textit{Service controller} berfungsi untuk mengatur \textit{resource service} seperti \textit{load balancer}
                    yang disediakan oleh \textit{cloud provider} yang digunakan.
          \end{itemize}
\end{itemize}

%%%%%%%%%%%%%%%%%%%%%%%%%%%%%%%%%%%%%%%%%%%%%%%%%%%%%%%%%%%%%%%%%%%%%%%%%%%%%%%%%%%%%%%%%%%%
% Komponen worker node
%%%%%%%%%%%%%%%%%%%%%%%%%%%%%%%%%%%%%%%%%%%%%%%%%%%%%%%%%%%%%%%%%%%%%%%%%%%%%%%%%%%%%%%%%%%%
\vspace{0.5cm}
\subsection{Komponen Worker node} \label{subsec:Komponen Worker Node}
\textit{Worker node} adalah server yang menjalankan \textit{pod} yang ada pada \textit{deployment}.
Pada \textit{worker node} terdapat beberapa komponen yang berjalan. Komponen-komponen tersebut antara lain:
\begin{itemize}
    \item \textit{kubelet} \\
          \textit{kubelet} merupakan \textit{agent} yang berjalan pada \textit{worker node}.
          \textit{kubelet} bertugas untuk menjalankan \textit{pod} yang telah didefinisikan pada \textit{deployment}.
          \textit{kubelet} juga bertugas untuk melakukan \textit{monitoring} terhadap \textit{pod} yang berjalan pada \textit{worker node}.
          \textit{kubelet} juga bertugas untuk melakukan \textit{health checking} terhadap \textit{pod} yang berjalan pada \textit{worker node}.
          Jika \textit{pod} tersebut mengalami masalah maka \textit{kubelet} akan melakukan \textit{restart} terhadap \textit{pod} tersebut.
    \item \textit{container runtime} \\
          \textit{Container runtime} merupakan \textit{software} yang digunakan untuk menjalankan \textit{container}.
          \textit{Container runtime} yang digunakan pada \textit{worker node} mendukung beberapa jenis \textit{container runetime},
          contohnya: \textit{docker, containerd, cri-o} dan implementasi lain yang mengikuti standar \textit{kubernetes CRI} (\textit{Contianer Runtime Interface}).
    \item \textit{kube-proxy} \\
          \textit{Kube proxy} merupakan \textit{proxy} yang mengimplementasikan abstraksi dari
          service dalam kubernetes. \textit{Kube proxy} menjalankan berbagai aturan network
          untuk mengatur akses baik dari dalam atau luar cluster (\textit{egress}) ke dalam node (\textit{ingress}).
\end{itemize}

%%%%%%%%%%%%%%%%%%%%%%%%%%%%%%%%%%%%%%%%%%%%%%%%%%%%%%%%%%%%%%%%%%%%%%%%%%%%%%%%%%%%%%%%%%%%
% Komponen Workload
%%%%%%%%%%%%%%%%%%%%%%%%%%%%%%%%%%%%%%%%%%%%%%%%%%%%%%%%%%%%%%%%%%%%%%%%%%%%%%%%%%%%%%%%%%%%
\vspace{0.5cm}
\subsection{Kubernetes Workloads} \label{subsec:Komponen Worker Node}
\textit{Workloads} merupakan sebuah aplikasi yang berjalan pada \textit{Kubernetes}.
\textit{Workloads} bisa saja berupa satu atau beberapa aplikasi yang saling bekerja sama.
Sebuah workload dieksekusi didalam sebuah \textit{pods}. \textit{Pod} merupakan sebuah entitas yang berisi satu atau lebih \textit{container}.
\textit{Kubernetes} menyediakan beberapa \textit{built-in Workloads} seperti
\textit{Deployment, StatefulSet, DaemonSet, Job, CronJob, ReplicaSet}.
\begin{itemize}
    \item \textit{Pod} \\
          \textit{Pod} Pod merupakan object terkecil dalam Kubernetes.
          Sebuah pod berisikan satu atau lebih container yang memiliki sistem penyimpanan
          dan jaringan yang sama dan mempunyai spesifikasi untuk menjalankan container-container tersebut.
          Pada praktik umumnya sebuah \textit{pod} hanya berisi satu container. Namun, ada beberapa kasus
          dimana sebuah \textit{pod} dapat berisi lebih dari satu container seperti ketika kita ingin menjalankan
          sebuah \textit{sidecar container} yang berfungsi untuk membantu container utama dalam menjalankan
          suatu fungsi tertentu.
    \item \textit{Deployment} \\
          \textit{Deployment} merupakan configurasi deklaratif yang digunakan untuk mendefinisikan
          \textit{desired state} dari sebuah \textit{pod} atau \textit{replica set}.Pada sebuah \textit{deployment}
          kita dapat mendefinisikan berapa banyak \textit{pod} yang ingin kita jalankan, \textit{image} apa yang
          ingin kita gunakan, dan juga \textit{label} apa yang ingin kita gunakan. \textit{Deployment} juga
          dapat melakukan \textit{rolling update} terhadap \textit{pod} yang sudah ada.
          Jika sebuah \textit{Deployment} dihapus, maka \textit{pod} yang merupakan bagian dari \textit{Deployment} tersebut
          juga akan ikut terhapus.
    \item \textit{StatefulSet} \\
          \textit{StatefulSet} merupakan \textit{workload} yang digunakan untuk menjalankan aplikasi yang \textit{stateful}.
          Perbedaan utama antara \textit{StatefulSet} dan \textit{Deployment} adalah \textit{StatefulSet} akan menyimpan
          identitas untuk setiap \textit{pod}.
    \item \textit{DaemonSet} \\
          \textit{DaemonSet} merupakan \textit{workload} yang digunakan untuk memastikan bahwa setiap (atau sebagian) \textit{node}
          memiliki salinan \textit{pod}. Ketika node baru ditambahkan ke kluster, \textit{pod} juga akan ditambahkan ke node tersebut.
    \item \textit{Jobs} \\
          \textit{DaemonSet} merupakan \textit{workdload} yang digunakan untuk membuat \textit{pod} yang hanya berjalan
          untuk menyelesaikan sebuah \textit{task} tertentu. \textit{Job} akan dihapus setelah \textit{task} tersebut selesai.
          Contoh sederhana dari \textit{job} adalah melakukan komputasi untuk menentukan nilai $\pi$ hingga 1000 digit.
    \item \textit{CronJob} \\
          \textit{CronJob} merupakan \textit{workload} yang digunakan untuk menjalankan \textit{job} secara terjadwal.
          \textit{CronJob} akan menjalankan \textit{job} sesuai dengan jadwal yang telah ditentukan.
    \item \textit{ReplicaSet} \\
          \textit{ReplicaSet} merupakan \textit{workload} yang digunakan untuk membuat \textit{pod} yang memiliki spesifikasi yang sama.
          \textit{ReplicaSet} akan membuat \textit{pod} baru jika \textit{pod} yang sudah ada mengalami kegagalan.
\end{itemize}

\vspace{0.5cm}
\subsection{Kubernetes Services} \label{subsec:Kubernetes Services}
