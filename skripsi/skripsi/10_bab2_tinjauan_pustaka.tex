\renewcommand{\chaptername}{BAB}
%-----------------------------------------------------------------------------%
\chapter{TINJAUAN PUSTAKA}
%-----------------------------------------------------------------------------%
\vspace{4.5pt}
\setlength{\parskip}{0.5em}
\section{Penelitian Terdahulu} \label{sec:penelitian_terdahulu}

Dalam beberapa tahun terakhir, kebutuhan akan otomasi dan standarisasi proses
deployment pada aplikasi berbasis microservices di lingkungan Kubernetes
mengalami peningkatan yang signifikan. Hal ini sejalan dengan semakin
kompleksnya arsitektur aplikasi modern yang menuntut efisiensi, keamanan, serta
keandalan dalam pengelolaan siklus hidup perangkat lunak. Salah satu pendekatan
yang berkembang pesat untuk menjawab tantangan tersebut adalah GitOps, di mana
seluruh konfigurasi dan proses deployment didefinisikan secara deklaratif di
dalam repository Git yang berfungsi sebagai sumber kebenaran utama (single
source of truth). Argo CD merupakan salah satu tools GitOps yang banyak
diadopsi oleh industri maupun komunitas open source karena kemampuannya dalam
melakukan sinkronisasi otomatis antara repository Git dan cluster Kubernetes,
serta menyediakan fitur visualisasi status deployment secara real-time.

Penelitian yang dilakukan oleh Korhonen \cite{Korhonen2021} secara khusus
mengevaluasi penerapan Argo CD dalam manajemen layanan berbasis Kubernetes.
Dalam penelitian tersebut, Korhonen mengimplementasikan pipeline CI/CD dengan
menggunakan GitLab sebagai sumber repository dan Argo CD sebagai alat
continuous delivery pada cluster Kubernetes yang dibangun dengan MicroK8s.
Hasil penelitian menunjukkan bahwa Argo CD mampu meningkatkan konsistensi
konfigurasi antar lingkungan, memudahkan rollback saat terjadi kegagalan
deployment, serta mempercepat proses delivery aplikasi melalui mekanisme
automated synchronization. Selain itu, Argo CD juga dinilai mempermudah proses
audit dan monitoring perubahan konfigurasi karena seluruh riwayat perubahan
tercatat dengan baik di repository Git. Namun demikian, Korhonen juga menyoroti
pentingnya perencanaan arsitektur dan pengelolaan dependensi layanan agar tidak
terjadi konflik konfigurasi yang dapat menghambat proses deployment secara
keseluruhan.

Studi lain yang dilakukan oleh Sharma et al. \cite{Sharma2022} membandingkan
performa deployment antara Argo CD dan Flux sebagai tools GitOps pada
lingkungan multi-cluster Kubernetes. Penelitian ini menggunakan beberapa
parameter evaluasi seperti kecepatan deployment, kemudahan integrasi dengan
pipeline CI/CD, serta kemampuan visualisasi status aplikasi. Hasilnya, Argo CD
dinilai lebih unggul dalam hal user experience, khususnya pada fitur
visualisasi status deployment dan kemudahan integrasi dengan berbagai pipeline
CI/CD yang umum digunakan di industri. Di sisi lain, Flux memiliki keunggulan
pada fleksibilitas manajemen konfigurasi dan kemudahan dalam melakukan
kustomisasi pada skenario deployment yang kompleks. Kedua tools ini sama-sama
mampu mengurangi potensi human error, meningkatkan traceability, serta
mempercepat proses deployment aplikasi secara signifikan jika dibandingkan
dengan metode deployment manual atau konvensional.

Selain aspek fungsionalitas dan performa, isu keamanan dalam implementasi Argo
CD juga menjadi perhatian dalam beberapa penelitian. Kumar \cite{Kumar2023}
membahas secara mendalam tantangan keamanan yang dihadapi dalam penggunaan Argo
CD, terutama terkait pengelolaan secrets dan akses kontrol pada cluster
Kubernetes. Dalam penelitian tersebut, ditemukan bahwa praktik terbaik yang
dapat diterapkan untuk meminimalisir risiko kebocoran data adalah dengan
mengintegrasikan Argo CD dengan external secrets management seperti HashiCorp
Vault atau AWS Secrets Manager, serta menerapkan prinsip least privilege pada
setiap komponen yang terlibat dalam proses deployment. Selain itu, Kumar juga
merekomendasikan penerapan audit log yang komprehensif dan pemantauan akses
secara real-time untuk meningkatkan keamanan dan akuntabilitas sistem.

\begin{table}[H]
  \centering
  \begin{adjustbox}{width=\textwidth}
    \begin{tabular}{|c|p{4cm}|p{3cm}|p{2.5cm}|p{3cm}|c|}
      \hline
      \textbf{No} & \textbf{Insight}                                                                         & \textbf{Hasil}                                                                                                                                                        & \textbf{Metode}                                                                            & \textbf{Batasan}                                                                               & \textbf{No Kutipan} \\
      \hline
      1           & Evaluasi Argo CD sebagai alat continuous delivery berbasis GitOps.                       & Argo CD meningkatkan konsistensi konfigurasi, memudahkan rollback, mempercepat delivery, dan mempermudah audit.                                                       & Studi kasus implementasi pipeline CI/CD dengan GitLab dan MicroK8s                         & Perlu perencanaan arsitektur dan pengelolaan dependensi agar tidak terjadi konflik konfigurasi & \cite{Korhonen2021} \\
      \hline
      2           & Komparasi performa deployment antara Argo CD dan Flux pada multi-cluster Kubernetes.     & Argo CD unggul pada user experience dan visualisasi, Flux lebih fleksibel dalam manajemen konfigurasi. Keduanya meningkatkan traceability dan mengurangi human error. & Eksperimen komparatif pada beberapa parameter evaluasi (kecepatan, integrasi, visualisasi) & Hanya membandingkan dua tools GitOps utama, tidak membahas aspek keamanan secara mendalam      & \cite{Sharma2022}   \\
      \hline
      3           & Isu keamanan pada implementasi Argo CD, khususnya pengelolaan secrets dan akses kontrol. & Integrasi dengan external secrets management dan penerapan least privilege meningkatkan keamanan. Audit log dan monitoring real-time direkomendasikan.                & Studi literatur dan analisis praktik keamanan pada Argo CD                                 & Fokus pada keamanan, belum membahas performa atau integrasi pipeline secara detail             & \cite{Kumar2023}    \\
      \hline
    \end{tabular}
  \end{adjustbox}
  \caption{Rangkuman Penelitian Terdahulu}
  \label{tab:tinjauan-pustaka}
\end{table}

\section{Kubernetes}
Kubernetes, sering disingkat K8s, adalah sebuah platform open-source yang
portabel dan dapat diperluas, yang dirancang untuk mengelola workload dan
layanan yang dikontainerisasi. Awalnya dikembangkan oleh Google dan kini
dikelola oleh Cloud Native Computing Foundation (CNCF) \cite{kubernetes_2021},
Kubernetes telah menjadi standar industri untuk orkestrasi kontainer. Tujuan
utamanya adalah menyediakan platform yang mampu mengotomatisasi proses
deployment, penskalaan, dan operasi aplikasi dalam kontainer, sehingga
menyederhanakan kompleksitas pengelolaan aplikasi di berbagai lingkungan
komputasi

\section{Argo CD}
Argo CD merupakan salah satu perangkat GitOps terkemuka yang telah diadopsi
secara luas, baik di lingkungan industri maupun oleh komunitas \textit{open
  source}. Popularitasnya didorong oleh kemampuannya untuk melakukan sinkronisasi
secara otomatis antara repositori Git dengan klaster Kubernetes, seraya
menyajikan visualisasi status \textit{deployment} secara \textit{real-time}
\cite{ArgoCDDocs}. Sebagai proyek yang bernaung di bawah Cloud Native Computing
Foundation (CNCF), Argo CD menawarkan solusi \textit{continuous delivery} yang
bersifat deklaratif, dengan menjadikan Git sebagai satu-satunya sumber
kebenaran (\textit{single source of truth}) untuk seluruh konfigurasi aplikasi.

Arsitekturnya yang berbasis operator memungkinkan Argo CD untuk terus memantau
aplikasi yang berjalan dan secara aktif menyelaraskan keadaan aplikasi di
klaster (\textit{actual state}) agar selalu sesuai dengan konfigurasi yang
diinginkan (\textit{desired state}) di repositori Git. Beberapa fitur
unggulannya mencakup dukungan untuk lingkungan \textit{multi-tenant} dan
Role-Based Access Control (RBAC), kemampuan untuk melakukan \textit{rollback}
ke versi sebelumnya, serta integrasi dengan berbagai sistem autentikasi seperti
OAuth2.

\section{GitOps}
GitOps adalah pendekatan yang menggabungkan prinsip-prinsip Git (versi kontrol)
dengan DevOps (operasionalisasi). Dengan GitOps, konfigurasi dan kode sumber
aplikasi disimpan di repository Git, dan Argo CD (atau tools GitOps lainnya)
digunakan untuk sinkronisasi otomatis antara repository Git dan cluster
Kubernetes. Ini memungkinkan otomatisasi proses deployment dan meningkatkan
traceability, auditability, dan keandalan dalam siklus hidup perangkat lunak
\cite{Beetz2022}.

\section{Penelitian Lanjutan}
Berdasarkan tinjauan terhadap penelitian-penelitian terdahulu pada
\textbf{Tabel \ref{tab:tinjauan-pustaka}}, dapat disimpulkan bahwa penggunaan
Argo CD dalam workflow GitOps membawa dampak positif yang signifikan terhadap
efisiensi, keamanan, dan auditability proses deployment aplikasi berbasis
microservices di Kubernetes. Namun demikian, masih terdapat ruang penelitian
lebih lanjut terkait evaluasi komparatif antara metode pull-based deployment
(seperti Argo CD) dengan metode push-based, serta analisis mendalam mengenai
best practice implementasi Argo CD pada skala enterprise, khususnya dalam
konteks lingkungan produksi yang sangat dinamis dan kompleks. Oleh karena itu,
penelitian ini berfokus pada analisis dan evaluasi penggunaan Argo CD sebagai
operator GitOps pada Kubernetes, dengan tujuan memberikan rekomendasi strategis
bagi organisasi yang ingin mengadopsi otomasi deployment berbasis GitOps secara
optimal.

\newpage
