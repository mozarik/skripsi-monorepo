%
% Halaman Abstract
\phantomsection \addcontentsline{toc}{chapter}{ABSTRAK}
\chapter*{ABSTRAK}

\begin{singlespace}
	Penelitian ini membahas implementasi ArgoCD dengan pendekatan GitOps pada lingkungan Kubernetes yang berjalan di atas infrastruktur bare-metal menggunakan Proxmox VE dan Talos OS. Tujuan penelitian ini adalah untuk mengevaluasi efektivitas ArgoCD dalam mengotomatisasi proses deployment aplikasi berbasis microservices dengan prinsip GitOps. Metode yang digunakan meliputi studi literatur, perancangan arsitektur, implementasi, dan pengujian. Hasil penelitian menunjukkan bahwa ArgoCD berhasil diimplementasikan dengan baik pada lingkungan Kubernetes, menyediakan mekanisme sinkronisasi otomatis antara konfigurasi yang dideklarasikan di Git dengan status aktual di cluster. Integrasi dengan Cloudflare Tunnel memungkinkan akses aman ke aplikasi tanpa perlu membuka port firewall secara langsung. Hasil pengujian menunjukkan tingkat keandalan sistem mencapai 99.9\% dengan kemampuan rollback yang efektif dalam waktu kurang dari 1 menit. Penelitian ini membuktikan bahwa pendekatan GitOps dengan ArgoCD dapat diterapkan secara efektif di lingkungan non-cloud, memberikan keuntungan berupa peningkatan keamanan, auditabilitas, dan kemudahan dalam manajemen konfigurasi. \\[20pt]
	Kata Kunci: \textit{ArgoCD, GitOps, Kubernetes, Continuous Deployment, Cloudflare Tunnel}
\end{singlespace}

\newpage

\phantomsection \addcontentsline{toc}{chapter}{ABSTRACT}
\chapter*{ABSTRACT}

\begin{singlespace}
	This research discusses the implementation of ArgoCD with a GitOps approach in a Kubernetes environment running on bare-metal infrastructure using Proxmox VE and Talos OS. The objective of this study is to evaluate the effectiveness of ArgoCD in automating the deployment process of microservices-based applications using GitOps principles. The methodology includes literature study, architectural design, implementation, and testing. The results show that ArgoCD was successfully implemented in the Kubernetes environment, providing an automatic synchronization mechanism between the configuration declared in Git and the actual state in the cluster. Integration with Cloudflare Tunnel enables secure access to applications without the need to directly expose firewall ports. Testing results demonstrate a system reliability rate of 99.9\% with effective rollback capability in less than 1 minute. This research proves that the GitOps approach with ArgoCD can be effectively applied in non-cloud environments, offering benefits such as enhanced security, auditability, and ease of configuration management. \\[20pt]
	Keywords: \textit{ArgoCD, GitOps, Kubernetes, Continuous Deployment, Cloudflare Tunnel}
\end{singlespace}

\newpage
