\renewcommand{\chaptername}{BAB}
%-----------------------------------------------------------------------------%
\chapter{HASIL DAN PEMBAHASAN}
%-----------------------------------------------------------------------------%
\vspace{4.5pt}
\setlength{\parskip}{0.5em}
\section{Perancangan (Design)}\label{sec:bab4_perancangan}
Pada tahan pertama ini peneliti akan melakukan perancangan sebuah infrastruktur
dimana sistem Kubernetes dan ArgoCD akan dijalankan lalu microservice untuk
dilakukan simulasi implementasi pada ArgoCD. Semua rancangan akan menggunakan
visualisasi diagram secara garis besar (high-level) agar mudah dipahami.

\subsection{Perancangan Sistem Arsitektur Infrastruktur Mesin}
Sebelum dilakukan penelitian lebih lanjut pada implementasi ArgoCD pada
Kubernetes Cluster perlu dirancang terlebih dahulu arsitektur infrastruktur
mesin yang akan menjadi tempat instlasi Kubernetes Cluster. Peneliti
menggunakan Proxmox VE (Virtual Environment) yang didalam nya terdapat Virtual
Machine berupa Talos OS Linux yang siap dilakukan instalasi Kubernetes Cluster.
\begin{figure}[H]
  \centering
  \fcolorbox{black}{white}{\includegraphics[width=0.9\textwidth]{figures/proxmox-cluster-design-1.png}}
  \caption{Arsitektur Infrastruktur High Level}
  \label{fig:arsitektur_infrastruktur_mesin}
\end{figure}
Pada \textbf{Gambar \ref{fig:arsitektur_infrastruktur_mesin}} menunjukan
arsitektur infrastruktur mesin secara high level dimana terdapat 2 cluster
proxmox yang di \textit{joined} agar bisa saling terhubung satu sama lain.
Akses terhadap proxmox tersebut dapat dilakukan menggunakan port 8006 dengan
menggunakan browser atau dengan kata lain bisa diakses dari
https://192.168.0.194:8006/ atau https://192.168.0.195:8006/
\begin{figure}[H]
  \centering
  \fcolorbox{black}{white}{\includegraphics[width=0.9\textwidth]{figures/proxmox-cluster.png}}
  \caption{Arsitektur Virtual Machine Talos OS pada Proxmox Cluster}
  \label{fig:arsitektur_virtual_machine_talos_os}
\end{figure}

Setelah dilakukan \textit{provisioning} pada proxmox tersebut akan dilakukan
instalasi Talos OS pada node tersebut menggunakan dengan memanfaatkan
\textit{virtual machine} yang ada pada proxmox. Pada \textbf{Gambar
  \ref{fig:arsitektur_virtual_machine_talos_os}} merupakan arsitektur pada
kubernetes cluster itu sendiri dimana sistem virtual machine yang didalamnya
terdapat komponen Talos OS dan komponen kubernetes. Kubernetes Cluster dapat
diakses menggunakan address 192.168.0.253:6443 menggunakan kubectl didalam
address tersebut terdapat 2 node Talos OS yang masing-masing memiliki komponen
kubernetes node.

\subsection{Perancangan Arsitektur ArgoCD pada Kubernetes Cluster}
Pada bagian sebelumnya peneliti sudah memaparkan arsitektur infrastruktur mesin
dimana Kubernetes cluster akan berjalan. Pada tahap ini peneliti akan
memaparkan arsitektur pada kubernetes cluster nya itu sendiri dimana ArgoCD
akan berjalan.
\begin{figure}[H]
  \centering
  \fcolorbox{black}{white}{\includegraphics[width=0.9\textwidth]{figures/kubernetes-cluster-1-new.png}}
  \caption{Arsitektur Kubernetes High Level}
  \label{fig:arsitektur_kubernetes_inside_proxmox}
\end{figure}

Pada \textbf{Gambar \ref{fig:arsitektur_kubernetes_inside_proxmox}}
mengambarkan arsitektur kubernetes nya secara high level dimana terdapat
komponen ArgoCD dan Cloudflare Tunnel yang bertujuan agar server microservice
bisa diakses pada world wide web (internet). Secara garis besar sebuah service
digambarkan sebagai app yang terdeploy pada sebuah namespace yang terdapat pada
kubernetes cluster. Lalu \textit{ingress} akan digunakan untuk mengakses
service tersebut dari luar cluster dengan menggunakan domain yang sudah
terdaftar pada cloudflare.

\subsection{Perancangan Arsitektur Microservice pada Kubernetes Cluster}
Pada bagian ini peneliti akan memaparkan contoh design arsitektur microservice
yang akan dijadikan acuan untuk dilakukan implementasi pada kubernetes cluster.
Nantinya service service ini akan berdiri sendiri dan mempunya deployment
sendiri pada kubernetes cluster. Deployment service tersebut akan dideploy
menggunakan ArgoCD agar bisa terdeploy kedalam kubernetes cluster.
\begin{figure}[H]
  \centering
  \fcolorbox{black}{white}{\includegraphics[width=0.9\textwidth]{figures/microservice-arsitektur.drawio.png}}
  \caption{Arsitektur Microservice High Level}
  \label{fig:arsitektur_microservice_high_level}
\end{figure}

Pada \textbf{Gambar \ref{fig:arsitektur_microservice_high_level}} menggambarkan
arsitektur microservice yang akan dideploy pada kubernetes cluster menggunakan
Argo CD. Microservice tersebut terdiri dari beberapa service yaitu service
frontend yang akan berfungsi sebagai interface untuk user dan service backend
yang akan diakses. Terdapat juga service auth yang akan berfungsi sebagai
autentikasi dan otorisasi user sebelum bisa mengakses service backend yaitu
color service dan prime generator service. Auth service sendiri bersifat
stateless komponen auth db pada arsitektur digambarkan untuk menyimpan data
user yang akan diakses oleh service auth. Color service dan prime generator
service sendiri juga bersifat stateless untuk mempermudah implementasi
microservice tersebut.

\section{Pengkodean (Coding) / Implementasi}\label{sec:bab4_implementasi}
Tahap ini peneliti akan menjabarkan secara rinci pengkodean/implementasi
rancangan sistem yang sudah dirancang sebelumnya pada bagian
\textbf{\ref{sec:bab4_perancangan} Perancangan (Design)}. Tahap implementasi
akan dijelaskan secara berurutan sesuai dengan urutan yang dijabarkan
sebelumnya pada bagian \textbf{\ref{sec:bab4_perancangan} Perancangan
  (Design)}.

\subsection{Implementasi Sistem Infrastruktur Mesin}
Untuk implementasi infrastruktur mesin. Peneliti akan melakukan instalasi
proxmox pada 2 mesin dengan arsitektur CPU X86. \textbf{Tabel
  \ref{tab:node-kubernetes}} merupakan spesifikasi node yang akan digunakan pada
penelitian ini.

\begin{table}[H]
  \centering
  \begin{tabular}{|l|l|l|c|c|c|}
    \hline
    \textbf{Node} & \textbf{Host} & \textbf{OS}    & \textbf{CPU (Core)} & \textbf{RAM} & \textbf{Kubernetes Version} \\
    \hline
    master-1      & 192.168.0.194 & Talos OS 1.9.5 & 4                   & 12 GB        & v1.28.0                     \\
    \hline
    worker-1      & 192.168.0.195 & Talos OS 1.9.5 & 4                   & 12 GB        & v1.28.0                     \\
    \hline
  \end{tabular}
  \caption{Spesifikasi Node Kubernetes}
  \label{tab:node-kubernetes}
\end{table}
\begin{figure}[H]
  \centering
  \includegraphics[width=0.8\textwidth]{figures/proxmox-install-rufus-1.jpg}
  \caption{Pembuatan Bootable Proxmox Menggunakan Rufus}
  \label{fig:proxmox_rufus}
\end{figure}
Pertama peneliti akan download ISO file atau installer proxmox yang terdapat di
link ini \verb|https://enterprise.proxmox.com/iso/proxmox-ve_8.4-1.iso|. Setelah
itu peneliti memerlukan sebuah Flashdisk atau media untuk bootable ISO
tersebut disini peneliti menggunakan tools yang bernama Rufus seperti pada \textbf{Gambar \ref{fig:proxmox_rufus}}. Setelah itu kita perlu mengganti bootable menu yang mengarah pada flashdisk
atau media yang kita gunakan untuk instalasi proxmox ketika booting BIOS pada
mesin yang digunakan. Setelah itu mesin akan restart dan akan masuk pada tampilan
seperti pada \textbf{Gambar \ref{fig:proxmox_install_ui}}.

\begin{figure}[H]
  \centering
  \fcolorbox{black}{white}{\includegraphics[width=0.8\textwidth]{figures/proxmox-install-ui-2.jpg}}
  \caption{Tampilan Awal Instalasi Proxmox VE}
  \label{fig:proxmox_install_ui}
\end{figure}

Selanjutnya kita tinggal mengikuti apa yang diarahkan secara default oleh user
interface yang ditampilkan hingga mesin akan restart dan berada pada tampilan
seperti pada \textbf{Gambar \ref{fig:proxmox_terminal}}. Pada tampilan layar
tersebut terdapat alamat server yang bisa kita gunakan untuk akses user
interface melalui browser.
\begin{figure}[H]
  \centering
  \fcolorbox{black}{white}{\includegraphics[width=0.8\textwidth]{figures/proxmox-install-terminal-3.jpg}}
  \caption{Tampilan Terminal Setelah Instalasi Proxmox Selesai}
  \label{fig:proxmox_terminal}
\end{figure}

Akses UI pada komputer yang ada pada network yang sama dengan Proxmox melalui
web alamat server yang terlihat pada gambar \textbf{Gambar
  \ref{fig:proxmox_terminal}} maka akan muncul tampilan seperti pada
\textbf{Gambar \ref{fig:proxmox_webui}}. Peneliti akan mengulang implementasi
ini untuk mesin kedua dengan step yang sama seperti pada mesin pertama.
\begin{figure}[H]
  \centering
  \fcolorbox{black}{white}{\includegraphics[width=0.9\textwidth]{figures/proxmox-install-web-ui-4.jpg}}
  \caption{Tampilan Web Interface Proxmox VE}
  \label{fig:proxmox_webui}
\end{figure}

Pada tahap selanjutnya akan dilakukan instalasi Talos OS yang akan di-install
menggunakan VM yang ada pada Proxmox VE. Pertama yang dilakukan adalah
mengunduh ISO file Talos OS di sini.

\url{https://github.com/siderolabs/talos/releases/download/v1.9.5/metal-amd64.iso}

\begin{figure}[H]
  \centering
  \fcolorbox{black}{white}{\includegraphics[width=0.8\textwidth]{figures/talos-install-1.jpg}}
  \caption{Proses Download ISO Talos OS}
  \label{fig:talos_download}
\end{figure}

Url tersebut lalu didownload melalui proxmox agar tersimpan didalam proxmox
seperti pada \textbf{Gambar \ref{fig:talos_download}}. Lalu tahap selanjutnya
adalah melakukan instalasi VM Talos OS pada proxmox. Tahap ini adalah bagian
instalasi VM Talos OS. Peneliti melakukan instalasi pada proxmox VE melalui
interface web. Tahap ini akan dilakukan pada 2 mesin berbeda pada proxmox.
Tahap-tahap instalasi Talos OS tersebut dijabarkan secara berurutan yang
ditunjukkan pada \textbf{Gambar \ref{fig:talos_install_1} -
  \ref{fig:talos_install_9}}

\begin{figure}[!htbp]
  1. Klik Create VM lalu isikan VM ID dan Name untuk VM Talos OS
  \centering
  \includegraphics[width=1\textwidth]{figures/talos-install-2.jpg}
  \caption{Instalasi Talos OS 1}
  \label{fig:talos_install_1}
\end{figure}
\begin{figure}[!htbp]
  2. Pada bagian OS. Pilih ISO Talos OS yang baru saja diunduh sebelum nya
  \centering
  \includegraphics[width=1\textwidth]{figures/talos-install-3.jpg}
  \caption{Instalasi Talos OS 2}
  \label{fig:talos_install_2}
\end{figure}
\begin{figure}[!htbp]
  3. Pada bagian System. cukup next
  \centering
  \includegraphics[width=1\textwidth]{figures/talos-install-4.jpg}
  \caption{Instalasi Talos OS 3}
  \label{fig:talos_install_3}
\end{figure}
\begin{figure}[!htbp]
  4. Pada bagian Disk. Cuku merubah bagian Disk size menjadi 100
  \centering
  \includegraphics[width=1\textwidth]{figures/talos-install-5.jpg}
  \caption{Instalasi Talos OS 4}
  \label{fig:talos_install_4}
\end{figure}
\begin{figure}[!htbp]
  5. Pada bagian CPU. Cukup merubah cores menjadi 2 atau 4 sesuai spesifikasi mesin yang diinginkan
  \centering
  \includegraphics[width=1\textwidth]{figures/talos-install-6.jpg}
  \caption{Instalasi Talos OS 5}
  \label{fig:talos_install_5}
\end{figure}
\begin{figure}[!htbp]
  6. Pada bagian memory. Isikan memory sesuai yang dinginkan dengan minimal 8192 MiB
  \centering
  \includegraphics[width=1\textwidth]{figures/talos-install-7.jpg}
  \caption{Instalasi Talos OS 6}
  \label{fig:talos_install_6}
\end{figure}
\begin{figure}[!htbp]
  7. Pada bagian Network. Silahkan klik next
  \centering
  \includegraphics[width=1\textwidth]{figures/talos-install-8.jpg}
  \caption{Instalasi Talos OS 7}
  \label{fig:talos_install_7}
\end{figure}
\begin{figure}[!htbp]
  8. Pada bagian confirm. Centang "Start After Created" lalu Finish
  \centering
  \includegraphics[width=1\textwidth]{figures/talos-install-9.jpg}
  \caption{Instalasi Talos OS 8}
  \label{fig:talos_install_8}
\end{figure}
\begin{figure}[!htbp]
  10. Ketika Talos OS Sudah berjalan terminal Talos OS akan tampak seperti ini
  \centering
  \includegraphics[width=1\textwidth]{figures/talos-install-10.jpg}
  \caption{Instalasi Talos OS 9}
  \label{fig:talos_install_9}
\end{figure}
\newpage
\subsubsection{Implementasi Instalasi Kubernetes pada Talos OS}
Tahap ini adalah tahap untuk instalasi Kubernetes pada Talos OS. Disini
peneliti akan menggunakan deklaratif yaml yang akan mengautomatisasi instalasi
kubernetes pada Talos OS. Instalasi Kubernetes pada Talos OS sendiri mengikuti
panduan dari pedoman yang terdapat pada website Talos OS
\url{https://www.talos.dev/v1.9.5/introduction/getting-started/}. \textbf{Tabel
  \ref{tab:yaml-code}} menunjukkan konfigurasi script instalasi Kubernetes
cluster pada Talos OS terdapat beberapa langkah yang akan dijalankan yaitu
dengan melakukan generate secret, generate config, generate command apply,
generate command bootstrap, dan generate command kubeconfig yang akan digunakan
sebagai konfigurasi dasar untuk instalasi Kubernetes cluster pada Talos OS
hasil dari generate config tersebut ditunjukan pada \textbf{Tabel
  \ref{tab:yaml-code-2}}.

\begin{table}[!htbp]
  \begin{lstlisting}[language=yaml, basicstyle=\footnotesize\ttfamily]
version: '3'
tasks:
  talos:
    desc: Bootstrap the Talos cluster
    dir: '{{.TALOS_DIR}}'
    cmds:
      - '[ -f talsecret.sops.yaml ] || talhelper gensecret | sops --filename-override talos/talsecret.sops.yaml --encrypt /dev/stdin > talsecret.sops.yaml'
      - talhelper genconfig
      - talhelper gencommand apply --extra-flags="--insecure" | bash
      - until talhelper gencommand bootstrap | bash; do sleep 10; done
      - until talhelper gencommand kubeconfig --extra-flags="{{.ROOT_DIR}} --force" | bash; do sleep 10; done
\end{lstlisting}
  \caption{Konfigurasi script instalasi Kubernetes cluster pada Talos OS}
  \label{tab:yaml-code}
\end{table}
\begin{table}[!htbp]
  \begin{lstlisting}[language=yaml, basicstyle=\footnotesize\ttfamily]
clusterName: kubernetes
endpoint: https://192.168.0.250:6443
...
nodes:
  - hostname: "kube-cp-1"
    ipAddress: "192.168.0.200"
    installDisk: "/dev/sda"
    machineSpec:
    controlPlane: true
    networkInterfaces:
      - deviceSelector:
          hardwareAddr: "de:ad:be:ef:00:01"
        addresses:
          - "192.168.0.200/24"
        routes:
          - network: "0.0.0.0/0"
            gateway: "192.168.0.1"
        mtu: 1500
        vip:
          ip: "192.168.0.250"
  - hostname: "kube-worker-1"
    ipAddress: "192.168.0.201"
    installDisk: "/dev/sda"
    machineSpec:
    controlPlane: false
    networkInterfaces:
      - deviceSelector:
          hardwareAddr: "de:ad:be:ef:00:02"
        dhcp: false
        addresses:
          - "192.168.0.201/24"
        routes:
          - network: "0.0.0.0/0"
            gateway: "192.168.0.1"
        mtu: 1500
\end{lstlisting}
  \caption{Konfigurasi script instalasi Kubernetes cluster pada Talos OS}
  \label{tab:yaml-code-2}
\end{table}
\newpage

Setelah dilakukan instalasi Kubernetes cluster pada Talos OS tampilan terminal
akan menampilkan nama cluster dan juga sudah tidak menampilan status Stage:
Maintenance melainkan sudah menampilkan Kubelet: Healthy seperti \textbf{Gambar
  \ref{fig:talos_install_11}}.

\begin{figure}[!ht]
  \centering
  \includegraphics[width=1\textwidth]{figures/talos-install-11.jpg}
  \caption{Tampilan Talos OS setelah instalasi kubernetes cluster}
  \label{fig:talos_install_11}
\end{figure}

\newpage

Setelah instalasi kubernetes pada Talos OS dilakukan maka kita dapat mengakses
kubernetes cluster tersebut menggunakan cli kubectl. Sebagai contoh
\textbf{Gambar \ref{fig:echo_web}} adalah aplikasi dummy yang menampilkan
informasi header pada browser pada \url{echo.zeinfahrozi.my.id}.

\begin{figure}[!ht]
  \centering
  \includegraphics[width=1\textwidth]{figures/echo-web.jpg}
  \caption{Tampilan echo.zeinfahrozi.my.id pada browser}
  \label{fig:echo_web}
\end{figure}

\subsection{Implementasi ArgoCD pada Kubernetes Cluster}
Tahap ini merupakan instalasi instance ArgoCD itu sendiri pada kubernetes
cluster yang sudah dibuat sebelumnya. ArgoCD dipasang menggunakan Helm chart
dengan konfigurasi kustom yang disesuaikan dengan kebutuhan lingkungan
produksi. Berikut adalah contoh perintah untuk menginstal ArgoCD menggunakan
Helm:

\begin{table}[H]
  \centering
  \begin{minipage}{0.95\linewidth}
    \begin{lstlisting}[language=bash, basicstyle=\footnotesize\ttfamily]
# Menambahkan repo ArgoCD
helm repo add argo https://argoproj.github.io/argo-helm
helm repo update

# Membuat namespace untuk ArgoCD
kubectl create namespace argocd

# Menginstal ArgoCD dengan Helm
helm upgrade --install argocd argo/argo-cd \
  --namespace argocd \
  --values values.sops.yaml \
  --wait
    \end{lstlisting}
  \end{minipage}
  \caption{Perintah Instalasi ArgoCD menggunakan Helm}
  \label{tab:install-argocd}
\end{table}
Pada \textbf{Table \ref{tab:install-argocd}} menunjukkan perintah instalasi
ArgoCD menggunakan Helm charts dengan nilai-nilai kustom yang didefinisikan
dalam file \texttt{values.sops.yaml} yang ada pada \textbf{Table \ref{tab:values-sops-yaml}}. Beberapa konfigurasi penting yang diterapkan:

\begin{enumerate}[label=\alph*.]
  \item ArgoCD diakses melalui domain \texttt{argo.zeinfahrozi.my.id}
  \item Mode \texttt{insecure} diaktifkan untuk pengembangan
  \item Fitur status badge diaktifkan untuk memantau status aplikasi
  \item Dukungan untuk multiple value files dengan skema yang berbeda
  \item Eksklusi sumber daya tertentu seperti CiliumIdentity dari manajemen ArgoCD
\end{enumerate}

\begin{table}[H]
  \centering
  \begin{minipage}{0.95\linewidth}
    \begin{lstlisting}[language=yaml, basicstyle=\footnotesize\ttfamily]
# values.sops.yaml
crds:
  install: true

global:
  domain: argo.zeinfahrozi.my.id

configs:
  params:
    server.insecure: true
  cm:
    statusbadge.enabled: true
    kustomize.buildOptions: --enable-alpha-plugins --enable-exec
    helm.valuesFileSchemes: >-
      secrets+gpg-import,secrets+gpg-import-kubernetes,
      secrets+age-import,secrets+age-import-kubernetes,
      secrets,secrets+literal,https
    resource.exclusions: |
      - apiGroups:
          - cilium.io
        kinds:
          - CiliumIdentity
        clusters:
          - "*"
    \end{lstlisting}
  \end{minipage}
  \caption{Contoh konfigurasi \texttt{values.sops.yaml} untuk ArgoCD}
  \label{tab:values-sops-yaml}
\end{table}

Setelah ArgoCD terinstal, aplikasi-aplikasi dapat dikelola menggunakan GitOps.
Setiap aplikasi didefinisikan sebagai kustom resource Kubernetes yang
mereferensikan repositori Git yang berisi manifest Kubernetes. ArgoCD akan
secara otomatis melakukan sinkronisasi antara status yang diinginkan (yang
didefinisikan di Git) dengan status aktual di cluster. Beberapa aspek keamanan
yang diterapkan pada instalasi ArgoCD ini antara lain:

\begin{enumerate}[label=\alph*.]
  \item Penggunaan HTTPS untuk akses web UI
  \item Integrasi dengan Dex untuk autentikasi
  \item Pembatasan akses berbasis peran (RBAC)
  \item Penyimpanan rahasia yang aman menggunakan SOPS
\end{enumerate}

Dengan konfigurasi ini, ArgoCD siap digunakan untuk mengelola aplikasi secara
deklaratif menggunakan prinsip GitOps, di mana semua perubahan konfigurasi
dilakukan melalui pull request dan version control system.

\subsubsection{Implementasi Cloudflare Tunnel}
Cloudflare Tunnel digunakan untuk mengekspos layanan dalam cluster ke internet
dengan aman tanpa perlu membuka port firewall. Berikut adalah langkah-langkah
implementasinya:

Sebelum memulai implementasi, pastikan beberapa hal berikut sudah disiapkan:
\begin{itemize}
  \item Memiliki akun Cloudflare
  \item Mendaftarkan domain yang akan digunakan
  \item Mengatur DNS di Cloudflare
\end{itemize}

Cloudflare Tunnel diinstal menggunakan ArgoCD dengan konfigurasi seperti yang
ditunjukkan pada \textbf{Tabel \ref{tab:cloudflared-argocd}}. Beberapa
konfigurasi penting dalam tabel tersebut antara lain:

\begin{itemize}
  \item \texttt{apiVersion} dan \texttt{kind}: Menentukan jenis resource Kubernetes yang akan dibuat, dalam hal ini adalah ArgoCD Application
  \item \texttt{metadata.name} dan \texttt{metadata.namespace}: Menentukan nama aplikasi (\texttt{cloudflared}) dan namespace tempat aplikasi akan di-deploy (\texttt{argo-system})
  \item \texttt{spec.project}: Menentukan project ArgoCD yang akan digunakan (\texttt{kubernetes})
  \item \texttt{spec.sources}: Mendefinisikan sumber konfigurasi aplikasi dari repository Git
        \begin{itemize}
          \item \texttt{repoURL}: URL repository Git yang berisi konfigurasi Cloudflare Tunnel
          \item \texttt{path}: Lokasi file konfigurasi dalam repository (\texttt{kubernetes/apps/network/cloudflared})
          \item \texttt{targetRevision}: Branch atau tag Git yang akan digunakan (\texttt{main})
        \end{itemize}
  \item \texttt{spec.destination}: Menentukan cluster dan namespace target untuk deployment
  \item \texttt{spec.syncPolicy.automated}: Mengaktifkan sinkronisasi otomatis dengan opsi:
        \begin{itemize}
          \item \texttt{prune: true}: Secara otomatis menghapus resource yang tidak lagi ada di Git
          \item \texttt{selfHeal: true}: Secara otomatis memperbaiki perubahan yang dibuat di luar ArgoCD
        \end{itemize}
\end{itemize}

\begin{table}[H]
  \centering
  \begin{minipage}{0.95\linewidth}
    \begin{lstlisting}[language=yaml, basicstyle=\footnotesize\ttfamily]
apiVersion: argoproj.io/v1alpha1
kind: Application
metadata:
  name: cloudflared
  namespace: argo-system
spec:
  project: kubernetes
  sources:
    - repoURL: "https://github.com/mozarik/zein-home-lab.git"
      path: kubernetes/apps/network/cloudflared
      targetRevision: main
  destination:
    name: in-cluster
    namespace: network
  syncPolicy:
    automated:
      prune: true
      selfHeal: true
    \end{lstlisting}
  \end{minipage}
  \caption{Konfigurasi ArgoCD untuk instalasi Cloudflare Tunnel}
  \label{tab:cloudflared-argocd}
\end{table}

Setelah terinstal, Cloudflare Tunnel perlu dikonfigurasi untuk meneruskan lalu
lintas ke layanan dalam cluster. Konfigurasi dasar untuk mengekspos ArgoCD
melalui Cloudflare Tunnel dapat dilihat pada \textbf{Tabel
  \ref{tab:cloudflared-config}}. Berikut penjelasan konfigurasi utamanya:

\begin{itemize}
  \item \texttt{tunnel}: ID unik yang mengidentifikasi tunnel Cloudflare. Nilai \texttt{<tunnel-id>} diganti dengan ID tunnel yang didapat dari Cloudflare.
  \item \texttt{credentials-file}: Lokasi file kredensial yang berisi token otentikasi untuk mengakses akun Cloudflare.
  \item \texttt{ingress}: Mendefinisikan aturan routing untuk lalu lintas yang masuk:
        \begin{itemize}
          \item Aturan pertama mengarahkan permintaan dengan hostname
                \texttt{argo.zeinfahrozi.my.id} ke service ArgoCD di dalam cluster
                (\texttt{http://argocd-server.argocd.svc.cluster.local:80})
          \item Aturan terakhir (\texttt{http\_status:404}) berfungsi sebagai fallback untuk
                menampilkan error 404 jika tidak ada aturan yang cocok
        \end{itemize}
\end{itemize}

\begin{table}[H]
  \centering
  \begin{minipage}{0.95\linewidth}
    \begin{lstlisting}[language=yaml, basicstyle=\footnotesize\ttfamily]
# config.yaml
tunnel: <tunnel-id>
credentials-file: /etc/cloudflared/credentials.json
ingress:
  - hostname: argo.zeinfahrozi.my.id
    service: http://argocd-server.argocd.svc.cluster.local:80
  - service: http_status:404
    \end{lstlisting}
  \end{minipage}
  \caption{Konfigurasi dasar Cloudflare Tunnel untuk ArgoCD}
  \label{tab:cloudflared-config}
\end{table}

Setelah konfigurasi di atas diterapkan, Cloudflare Tunnel akan secara otomatis
membuat koneksi aman dari jaringan Cloudflare ke cluster Kubernetes. Layanan
ArgoCD dapat diakses melalui domain \texttt{argo.zeinfahrozi.my.id} dengan
enkripsi TLS end-to-end. Semua lalu lintas akan melewati jaringan Cloudflare
sehingga meningkatkan keamanan dan performa akses.

\subsubsection{Konfigurasi Git Repository}
Git repository digunakan sebagai sumber kebenaran (source of truth) untuk
konfigurasi infrastruktur. Repository yang digunakan adalah
\url{https://github.com/mozarik/zein-home-lab}.

\begin{figure}[H]
  \centering
  \fcolorbox{black}{white}{\includegraphics[width=0.9\textwidth]{figures/pull-based-deployments.png}}
  \caption{Flow Pull Based Deployments}
  \label{fig:flow_pull_based_deployments}
\end{figure}

Pada \textbf{Gambar \ref{fig:flow_pull_based_deployments}} mengambarkan flow
pull based deployments yang akan digunakan pada penelitian ini. Pada flow ini
setiap perubahan yang terjadi pada repository akan di-sync dengan kubernetes
cluster melalui ArgoCD sebagai operator yang menjadi observer dan melakukan
sinkronisasi antara status yang diinginkan (yang didefinisikan di Git) dengan
status aktual di cluster.

\section{Implementasi Microservice}
Implementasi microservice pada penelitian ini terdiri dari beberapa service
yang ditulis menggunakan bahasa pemrograman Go/Golang. Service yang dibuat akan
dilakukan containerization menggunakan Docker agar dapat digunakan sebagai
image pada kubernetes cluster. Setelah containerization selesai, image tersebut
akan dideploy pada kubernetes cluster menggunakan ArgoCD.

Terdapat 4 Service yang akan diimplementasikan yaitu UI Service (Entrypoint),
Auth Service, Color Service, dan Prime Generator Service. UI Service akan
berfungsi sebagai entrypoint untuk user dan akan memanggil service lainnya
berdasarkan permintaan user. Auth Service akan berfungsi sebagai autentikasi
dan otorisasi user sebelum bisa mengakses service lainnya. Color Service akan
berfungsi sebagai service yang akan menghasilkan warna acak berdasarkan
permintaan user. Prime Generator Service akan berfungsi sebagai service yang
akan menghasilkan bilangan prima berdasarkan permintaan user dimana user
menginput batas digit bilangan prima yang diinginkan.

\subsection{Implementasi UI Service}
UI Service atau Frontend service akan menjadi entrypoint dari User agar dapat
mengakses service lainnya. UI Service akan menggunakan bahasa pemrograman
Go/Golang dengan menggunakan HTMX sebagai library untuk pengembangan UI. UI
Service terdiri dari beberapa halaman yaitu halaman login dan halaman dashboard
service yang digunakan untuk menampilkan data dari service lainnya.

\begin{table}[H]
  \centering
  \begin{minipage}{0.95\linewidth}
    \begin{lstlisting}[language=go, basicstyle=\footnotesize\ttfamily]
func serveHome(w http.ResponseWriter, r *http.Request) {
    html := fmt.Sprintf(`<!DOCTYPE html>
<html>
<head>
    <title>HTMX Frontend (Created By Zein)</title>
    <script src="https://unpkg.com/htmx.org@1.9.2"></script>
</head>
<body>
    <h1>Welcome</h1>
    <p><em>Version: %s</em></p>
    <a href="/login">Login to access protected page</a>
</body>
</html>`, version)
    w.Header().Set("Content-Type", "text/html")
    fmt.Fprint(w, html)
}

func serveLogin(w http.ResponseWriter, r *http.Request) {
    html := fmt.Sprintf(`<!DOCTYPE html>
<html>
<head>
    <title>Login</title>
    <script src="https://unpkg.com/htmx.org@1.9.2"></script>
</head>
<body>
    <h2>Login</h2>
    <p><em>Version: %s</em></p>
    <form hx-post="/do_login" hx-target="#login_result" hx-swap="innerHTML">
        <input type="text" name="username" placeholder="Username" required><br>
        <input type="password" name="password" placeholder="Password" required><br>
        <button type="submit">Login</button>
    </form>
    <div id="login_result"></div>
</body>
</html>`, version)
    w.Header().Set("Content-Type", "text/html")
    fmt.Fprint(w, html)
}
    \end{lstlisting}
  \end{minipage}
  \caption{Implementasi UI Service Halaman Login}
  \label{tab:implementasi-ui-service-login}
\end{table}
Pada \textbf{Table \ref{tab:implementasi-ui-service-login}} menampilkan bagian
code untuk halaman login UI Service. Halaman ini akan tampil juga menjadi
halama home UI Service. Pada Gambarxxx menampilkan halaman home sekaligus login page pada UI Service
dimana user dapat login menggunakan username dan password yang telah diatur
pada service Auth Service.
% TODO: Masukan gambar UI service login lalu jelaskan

\begin{table}[H]
  \centering
  \begin{minipage}{0.95\linewidth}
    \begin{lstlisting}[language=go, basicstyle=\footnotesize\ttfamily]
      func handleLogin(w http.ResponseWriter, r *http.Request) {
        username := r.FormValue("username")
        password := r.FormValue("password")
    
        req, err := http.NewRequest("POST", authServiceURL, nil)
        if err != nil {
            http.Error(w, "Internal error", http.StatusInternalServerError)
            return
        }
        req.SetBasicAuth(username, password)
        resp, err := http.DefaultClient.Do(req)
        if err != nil || resp.StatusCode != http.StatusOK {
            fmt.Fprint(w, "<span style='color:red'>Login failed</span>")
            return
        }
    
        http.SetCookie(w, &http.Cookie{
            Name:  "session_user",
            Value: username,
            Path:  "/",
        })
        fmt.Fprintf(w, `<span style="color:green">Login successful! <a href="/protected">Go to protected page</a></span><br><em>Version: %s</em>`, version)
    }
    \end{lstlisting}
  \end{minipage}
  \caption{Implementasi Algoritma Login UI Service}
  \label{tab:implementasi-algoritma-login-ui-service}
\end{table}
Pada \textbf{Table \ref{tab:implementasi-algoritma-login-ui-service}} menampilkan bagian
code untuk algoritma login UI Service. Algoritma ini akan melakukan autentikasi
user menggunakan username dan password yang diinputkan oleh user dan akan
mengirimkan request ke service Auth Service untuk memverifikasi autentikasi
user.

\begin{table}[H]
  \centering
  \begin{minipage}{0.95\linewidth}
    \begin{lstlisting}[language=go, basicstyle=\footnotesize\ttfamily]
      func serveProtected(w http.ResponseWriter, r *http.Request) {
        cookie, err := r.Cookie("session_user")
        if err != nil || cookie.Value == "" {
          http.Redirect(w, r, "/login", http.StatusFound)
          return
        }
      
        html := fmt.Sprintf(`<!DOCTYPE html>
      <html>
      <head>
          <title>Protected Page</title>
          <script src="https://unpkg.com/htmx.org@1.9.2"></script>
      </head>
      <body>
          <h2>Protected Page</h2>
          <p><em>Version: %s</em></p>
          <p>Welcome, %s!</p>
          <button hx-get="/get_color" hx-target="#color_result" hx-swap="innerHTML">Get Random Color</button>
          <div id="color_result"></div>
          <br>
          <a href="/">Back to Home</a>
      </body>
      </html>`, version, cookie.Value)
        w.Header().Set("Content-Type", "text/html")
        fmt.Fprint(w, html)
      }
      
      func handleGetColor(w http.ResponseWriter, r *http.Request) {
        resp, err := http.Get(colorServiceURL)
        if err != nil {
          http.Error(w, "Failed to get color", http.StatusInternalServerError)
          return
        }
        defer resp.Body.Close()
        color, _ := io.ReadAll(resp.Body)
        fmt.Fprintf(w, "<div style='color:%s'>Random Color: %s<br><em>Version: %s</em></div>", color, color, version)
      }
    \end{lstlisting}
  \end{minipage}
  \caption{Implementasi Dashboard Service}
  \label{tab:implementasi-algoritma-dashboard-service}
\end{table}

Pada \textbf{Table \ref{tab:implementasi-algoritma-dashboard-service}}
menampilkan implementasi halaman dashboard yang dilindungi (protected page)
beserta fungsionalitas untuk mendapatkan warna acak dari Color Service. Halaman
ini hanya dapat diakses oleh pengguna yang sudah login, yang ditandai dengan
adanya session cookie. Fungsi \texttt{serveProtected} akan mengecek keberadaan
cookie session dan menampilkan halaman dashboard jika valid, sementara
\texttt{handleGetColor} akan memanggil Color Service untuk mendapatkan warna
acak yang akan ditampilkan ke pengguna.

\subsection{Implementasi Auth Service}
Auth service sendiri mempunyai fungsionalitas untuk melakukan autentikasi user
agar dapat mengakses dashboard dari UI Service. Auth service akan memvalidasi
username dan password yang diinputkan oleh user dan akan mengirimkan response
ke UI Service untuk menampilkan halaman dashboard.

\begin{table}[H]
  \centering
  \begin{minipage}{0.95\linewidth}
    \begin{lstlisting}[language=go, basicstyle=\footnotesize\ttfamily]
    const (
        validUser = "user"
        validPass = "pass"
    )
    
    func main() {
        http.HandleFunc("/auth", func(w http.ResponseWriter, r *http.Request) {
            auth := r.Header.Get("Authorization")
            if !strings.HasPrefix(auth, "Basic ") {
                http.Error(w, "Unauthorized", http.StatusUnauthorized)
                return
            }
            payload, _ := base64.StdEncoding.DecodeString(
              strings.TrimPrefix(auth, "Basic ")
            )
            pair := strings.SplitN(string(payload), ":", 2)
            if len(pair) != 2 || pair[0] != validUser || pair[1] != validPass {
                http.Error(w, "Unauthorized", http.StatusUnauthorized)
                return
            }
            w.WriteHeader(http.StatusOK)
            w.Write([]byte("OK"))
        })
        http.ListenAndServe(":8080", nil)
    }
    \end{lstlisting}
  \end{minipage}
  \caption{Implementasi Auth Service}
  \label{tab:implementasi-auth-service}
\end{table}
Pada \textbf{Table \ref{tab:implementasi-auth-service}} menampilkan bagian
code untuk algoritma autentikasi user. Algoritma ini akan melakukan autentikasi
user menggunakan username dan password yang diinputkan oleh user dan akan
mengirimkan response ke UI Service untuk menampilkan halaman dashboard.

\subsection{Implementasi Color Service}
Color service sendiri mempunyai fungsionalitas untuk mendapatkan warna acak
yang akan ditampilkan ke pengguna. Pada \textbf{Table
  \ref{tab:implementasi-color-service}} menampilkan bagian code untuk algoritma
color service. Algoritma ini akan mendapatkan warna acak yang akan menjadi
response dari Color Service. Color Service akan mengirimkan response tersebut
ke UI Service yang akan menampilkan warna acak tersebut ke pengguna.

\begin{table}[H]
  \centering
  \begin{minipage}{0.95\linewidth}
    \begin{lstlisting}[language=go, basicstyle=\footnotesize\ttfamily]
    func main() {
        http.HandleFunc("/get_color", func(w http.ResponseWriter, r *http.Request) {
            colors := []string{"red", "green", "blue", "yellow", "purple", "orange"}
            color := colors[rand.Intn(len(colors))]
            fmt.Fprint(w, color)
        })
        http.ListenAndServe(":8080", nil)
    }
    \end{lstlisting}
  \end{minipage}
  \caption{Implementasi Color Service}
  \label{tab:implementasi-color-service}
\end{table}

\subsection{Implementasi Prime Generator Service}
Prime generator service sendiri mempunyai fungsionalitas untuk mendapatkan
bilangan prima berdasarkan input dari user. User akan melalukan input berupa
jumlah digit bilangan prima yang diinginkan dan akan mendapatkan response
berupa bilangan prima yang diinginkan. Pada \textbf{Table
  \ref{tab:implementasi-prime-generator-service}} menampilkan bagian code untuk
algoritma prime generator service. Algoritma ini akan mendapatkan bilangan
prima berdasarkan input dari user dan akan mengirimkan response tersebut ke UI
Service yang akan menampilkan bilangan prima tersebut ke pengguna.

\begin{table}[H]
  \centering
  \begin{minipage}{0.95\linewidth}
    \begin{lstlisting}[language=go, basicstyle=\footnotesize\ttfamily]
    func main() {
        http.HandleFunc("/get_prime", func(w http.ResponseWriter, r *http.Request) {
            digits := r.URL.Query().Get("digits")
            if digits == "" {
                http.Error(w, "Missing digits parameter", http.StatusBadRequest)
                return
            }
            prime := generatePrime(digits)
            fmt.Fprint(w, prime)
        })
        http.ListenAndServe(":8080", nil)
    }
    \end{lstlisting}
  \end{minipage}
  \caption{Implementasi Prime Generator Service}
  \label{tab:implementasi-prime-generator-service}
\end{table}

\section{Testing}
Setelah menyelesaikan tahap instalasi dan konfigurasi infrastruktur, langkah
selanjutnya adalah melakukan pengujian untuk memastikan seluruh komponen
berfungsi seperti yang diharapkan. Pengujian ini mencakup beberapa aspek
penting termasuk fungsionalitas Kubernetes cluster, integrasi ArgoCD, serta
alur kerja GitOps yang telah diterapkan. Melalui pengujian menyeluruh ini,
diharapkan dapat dievaluasi sejauh mana solusi yang dibangun mampu memenuhi
kebutuhan pengembangan dan operasional aplikasi secara efisien.

\subsection{Pengujian (Testing)}\label{sec:bab4_pengujian}
Pengujian black-box testing yang dilakukan terhadap sistem pada tahap ini
menggunakan metode validasi (validation). Metode validasi yang digunakan dalam
melakukan pengujian ini berfungsi untuk mengetahui valid atau tidaknya sebuah
fungsi dari sistem yang dibangun. Melakukan pengujian black-box dengan metode
validasi ini juga menentukan apakah sistem telah sesuai seperti apa yang
diinginkan oleh stakeholder pada tahap perencanaan. Pengujian black-box ini
dilakukan dengan jumlah test case sebanyak 15 (lima belas) yang mencakup
berbagai aspek fungsionalitas ArgoCD.

\subsection{Metodologi Pengujian}
Pengujian dilakukan dengan pendekatan black-box testing yang berfokus pada
fungsionalitas sistem tanpa memperhatikan struktur internal kode. Setiap test
case dirancang untuk memverifikasi fitur-fitur kunci dari ArgoCD dalam
mendukung alur kerja GitOps.

\subsection{Hasil Pengujian}\label{subsec:hasil_pengujian}
Berikut adalah daftar test case yang telah dilakukan beserta hasilnya:
\begin{table}[h]
  \centering
  \caption{Daftar Test Case Black-Box Testing}
  \label{tab:test-case}
  \small
  \begin{adjustbox}{width=\textwidth}
    \begin{tabular}{|p{0.8cm}|p{2.2cm}|p{4cm}|p{3.5cm}|p{1.2cm}|}
      \hline
      \textbf{Kode Uji} & \textbf{Nama Uji}         & \textbf{Kasus Uji}                                               & \textbf{Hasil Yang Diharapkan}                                     & \textbf{Status} \\
      \hline
      BT-001            & Login ke Dashboard ArgoCD & \begin{enumerate}[leftmargin=*,noitemsep,topsep=0pt,label=\arabic*.,widest=99]
                                                        \item Buka halaman login ArgoCD
                                                        \item Masukkan kredensial admin
                                                        \item Klik tombol login
                                                      \end{enumerate}  & Pengguna berhasil login dan diarahkan ke dashboard utama           & Valid                                                           \\ \hline
      BT-002            & Tambah Aplikasi Baru      & \begin{enumerate}[leftmargin=*,noitemsep,topsep=0pt,label=\arabic*.,widest=99]
                                                        \item Klik "New App"
                                                        \item Isi form dengan detail aplikasi
                                                        \item Klik "Create"
                                                      \end{enumerate} & Aplikasi baru berhasil dibuat dan muncul di daftar aplikasi        & Valid                                                            \\ \hline
      BT-003            & Sinkronisasi Otomatis     & \begin{enumerate}[leftmargin=*,noitemsep,topsep=0pt,label=\arabic*.,widest=99]
                                                        \item Buat perubahan pada file konfigurasi di repo Git
                                                        \item Push perubahan ke branch yang dimonitor
                                                      \end{enumerate} & ArgoCD mendeteksi perubahan dan melakukan sinkronisasi otomatis    & Valid                                                            \\ \hline
      BT-004            & Rollback Aplikasi         & \begin{enumerate}[leftmargin=*,noitemsep,topsep=0pt,label=\arabic*.,widest=99]
                                                        \item Pilih aplikasi
                                                        \item Klik "History and Rollback"
                                                        \item Pilih versi sebelumnya
                                                        \item Klik "Sync"
                                                      \end{enumerate} & Aplikasi berhasil di-rollback ke versi sebelumnya                  & Valid                                                            \\ \hline
      BT-005            & Validasi Status Kesehatan & \begin{enumerate}[leftmargin=*,noitemsep,topsep=0pt,label=\arabic*.,widest=99]
                                                        \item Deploy aplikasi dengan konfigurasi salah
                                                        \item Periksa status di dashboard
                                                      \end{enumerate} & Menampilkan status "Degraded" atau "Error" dengan pesan yang jelas & Valid                                                            \\ \hline
      BT-006            & Pencarian Aplikasi        & \begin{enumerate}[leftmargin=*,noitemsep,topsep=0pt,label=\arabic*.,widest=99]
                                                        \item Gunakan fitur search di dashboard
                                                        \item Masukkan nama aplikasi
                                                      \end{enumerate} & Menampilkan aplikasi yang sesuai dengan kata kunci pencarian       & Valid                                                            \\ \hline
      BT-007            & Filter Aplikasi           & \begin{enumerate}[leftmargin=*,noitemsep,topsep=0pt,label=\arabic*.,widest=99]
                                                        \item Gunakan filter berdasarkan status/kategori
                                                        \item Pilih filter tertentu
                                                      \end{enumerate} & Menampilkan aplikasi yang sesuai dengan filter yang dipilih        & Valid                                                            \\ \hline
    \end{tabular}
  \end{adjustbox}
\end{table}

\newpage
\clearpage
\begin{table}[H]
  \centering
  \caption{Daftar Test Case Black-Box Testing (Lanjutan)}
  \label{tab:test-case-2}
  \small
  \begin{adjustbox}{width=\textwidth}
    \begin{tabular}{|p{0.8cm}|p{2.2cm}|p{4cm}|p{3.5cm}|p{1.2cm}|}
      \hline
      \textbf{Kode Uji} & \textbf{Nama Uji}       & \textbf{Kasus Uji}                                                      & \textbf{Hasil Yang Diharapkan}                                      & \textbf{Status} \\
      \hline
      BT-008            & Manajemen Kluster       & \begin{enumerate}[leftmargin=*,noitemsep,topsep=0pt,label=\arabic*.,widest=99]
                                                      \item Tambah kluster baru
                                                      \item Verifikasi koneksi
                                                    \end{enumerate}        & Kluster baru terdaftar dan terhubung dengan status "Healthy"        & Valid                                                            \\ \hline
      BT-009            & Logout                  & \begin{enumerate}[leftmargin=*,noitemsep,topsep=0pt,label=\arabic*.,widest=99]
                                                      \item Klik profil pengguna
                                                      \item Pilih "Logout"
                                                    \end{enumerate}        & Pengguna berhasil logout dan diarahkan ke halaman login             & Valid                                                            \\ \hline
      BT-010            & Responsivitas UI        & \begin{enumerate}[leftmargin=*,noitemsep,topsep=0pt,label=\arabic*.,widest=99]
                                                      \item Akses dashboard dari berbagai perangkat (desktop, tablet, mobile)
                                                    \end{enumerate} & Tampilan UI menyesuaikan dengan ukuran layar                        & Valid                                                                   \\ \hline
      BT-011            & Notifikasi Sinkronisasi & \begin{enumerate}[leftmargin=*,noitemsep,topsep=0pt,label=\arabic*.,widest=99]
                                                      \item Lakukan sinkronisasi manual
                                                      \item Periksa notifikasi
                                                    \end{enumerate}        & Muncul notifikasi yang menampilkan status sinkronisasi              & Valid                                                            \\ \hline
      BT-012            & Error Handling          & \begin{enumerate}[leftmargin=*,noitemsep,topsep=0pt,label=\arabic*.,widest=99]
                                                      \item Masukkan URL repo Git yang tidak valid
                                                      \item Coba buat aplikasi
                                                    \end{enumerate}        & Menampilkan pesan error yang jelas tentang URL yang tidak valid     & Valid                                                            \\ \hline
      BT-013            & Ekspor Konfigurasi      & \begin{enumerate}[leftmargin=*,noitemsep,topsep=0pt,label=\arabic*.,widest=99]
                                                      \item Pilih aplikasi
                                                      \item Ekspor konfigurasi
                                                    \end{enumerate}        & File konfigurasi berhasil diunduh dalam format YAML                 & Valid                                                            \\ \hline
      BT-014            & Manajemen Izin          & \begin{enumerate}[leftmargin=*,noitemsep,topsep=0pt,label=\arabic*.,widest=99]
                                                      \item Buat pengguna dengan role terbatas
                                                      \item Verifikasi akses
                                                    \end{enumerate}        & Pengguna hanya dapat mengakses fitur sesuai role yang diberikan     & Valid                                                            \\ \hline
      BT-015            & Audit Log               & \begin{enumerate}[leftmargin=*,noitemsep,topsep=0pt,label=\arabic*.,widest=99]
                                                      \item Lakukan beberapa aksi di dashboard
                                                      \item Periksa halaman audit log
                                                    \end{enumerate}        & Semua aksi terekam dalam log dengan timestamp dan detail yang jelas & Valid                                                            \\ \hline
    \end{tabular}
  \end{adjustbox}
\end{table}

\subsection{Analisis Hasil Pengujian}\label{subsec:analisis_hasil}
Berdasarkan hasil pengujian yang telah dilakukan, dapat dianalisis bahwa
implementasi ArgoCD berhasil memenuhi kebutuhan dalam mendukung alur kerja
GitOps pada infrastruktur Kubernetes. Berikut adalah analisis mendalam dari
hasil pengujian:

\subsubsection{Ketahanan Sistem}
Sistem berhasil melewati semua skenario pengujian yang mencakup berbagai aspek
fungsionalitas ArgoCD. Hal ini menunjukkan bahwa arsitektur yang dirancang
telah memenuhi kebutuhan dasar dalam implementasi GitOps.

\subsubsection{Kesesuaian dengan Ekspektasi}
Dari 15 test case yang dilakukan, seluruhnya menunjukkan hasil yang sesuai
dengan ekspektasi. Ini menunjukkan bahwa ArgoCD dapat diandalkan untuk
mengelola aplikasi pada cluster Kubernetes dengan pendekatan GitOps.

\subsubsection{Keandalan Fitur Inti}
Fitur-fitur inti seperti sinkronisasi otomatis, rollback, dan manajemen
konfigurasi berfungsi dengan baik. Hal ini menjadi bukti bahwa ArgoCD dapat
diandalkan untuk keperluan continuous deployment dalam lingkungan produksi.

\subsubsection{Keterbatasan}
Meskipun semua test case berhasil, terdapat beberapa aspek yang memerlukan
perhatian lebih lanjut, seperti manajemen kredensial yang aman dan pengaturan
RBAC yang lebih ketat untuk keperluan produksi.

\subsubsection{Rekomendasi}
Berdasarkan hasil pengujian, berikut beberapa rekomendasi untuk pengembangan
selanjutnya:
\begin{enumerate}
  \item Implementasi mekanisme backup dan recovery yang lebih komprehensif
  \item Peningkatan pengujian keamanan dan penetrasi
  \item Pengembangan pipeline CI/CD yang lebih matang
  \item Implementasi monitoring dan alerting yang lebih baik
\end{enumerate}

% * Revisi Rumusan hasil rumusan di bab 4
% * Revisi terkait penulisan implementasi yang ada di github masukan di skripsi
% * Penjelasan terkait hasil implementasi service pada bab 4 (color, frontend, auth)

% * Tambahkan testing pada sisi service seperti (jaringan, waktu (speed))
% * gambar dan tabel dirujuk semua lalu diberikan border
% * dalam subbab perlu a,b,c bukan 1,2,3 (contoh di halaman 13)
% * gambar dijelaskan why nya
% * bab 2 tolong ditambahkan literatur nya dan ditampilkan penelitian terdahulu

% format penulisan
% * Penulisan sub bab harus melihat penjelasan dan kalimat yang diutarakan
% * perlu diperbaiki secara serius dalam uji microservice
% * uji fungsionalitas terhadap microservice pada bab 4
