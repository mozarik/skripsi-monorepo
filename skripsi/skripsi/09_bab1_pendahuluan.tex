\renewcommand{\chaptername}{BAB}
%-----------------------------------------------------------------------------%
\chapter{PENDAHULUAN}
%-----------------------------------------------------------------------------%

\vspace{4.5pt}
\setlength{\parskip}{0.5em}

\section{Latar Belakang} \label{sec:latar_belakang}
Sebelum digunakan oleh pengguna secara luas, sebuah perangkat lunak perlu melalui proses \textit{deployment} terlebih dahulu.
\textit{Deployment} perangkat lunak dapat didefinisikan sebagai akuisisi dan eksekusi sebuah perangkat lunak.
Proses ini biasanya dilakukan oleh seorang software deployer atau yang lebih dikenal
saat ini sebagai SRE (site reliability engineer) \cite{Lyu2007}.
Dengan demikian, \textit{deployment} dapat dikatakan sebagai aktivitas pasca-produksi
sebuah perangkat lunak sebelum digunakan oleh konsumen.
Proses \textit{deployment} perangkat lunak terdiri dari beberapa tahapan yang
saling berhubungan, seperti proses rilis perangkat lunak,
instalasi perangkat lunak ke dalam environment eksekusi,
dan aktivasi perangkat lunak \cite{Carzaniga1998}.

Pada saat melakukan \textit{deployment} sebuah sistem perangkat lunak,
beberapa aspek perlu diperhatikan,
seperti sub-komponen yang dibutuhkan atau \textit{package external},
serta \textit{resource} (hardware).
Untuk melakukan \textit{deployment} sub-komponen ini,
dibutuhkan sebuah konfigurasi yang mendeskripsikan versi sub-komponen
yang digunakan oleh perangkat lunak.
Dengan bahasa modern sekarang,
konfigurasi sub-komponen yang digunakan oleh main perangkat lunak biasanya sudah dapat
dihasilkan secara otomatis, contohnya adalah; \textit{go modules} \cite{go_mod} dalam bahasa Golang, \textit{requirement file} \cite{requirementPython} dalam bahasa Python, atau file \textit{RubySpec} \cite{rubySpec} pada bahasa Ruby.
\par

Terdapat beberapa karakteristik fundamental dalam \textit{deployment} perangkat lunak yang diidentifikasi oleh Alan Dearle \cite{Dearle2007}, yaitu:

\textbf{\textit{Release}} merupakan tahap penghubung antara proses \textit{deployment} dengan pengembangan perangkat lunak. Tahap ini mencakup seluruh operasi yang diperlukan untuk mempersiapkan sistem sebelum diserahkan kepada pengguna akhir. Aktivitas \textit{release} juga meliputi penentuan sumber daya (\textit{resource}) yang dibutuhkan agar sistem dapat beroperasi dengan optimal pada \textit{environment} yang ditargetkan.

Selanjutnya dilakukan \textit{packaging} terhadap sistem perangkat lunak. \textit{Package} yang dihasilkan harus memuat seluruh komponen yang dibutuhkan, termasuk deskripsi sistem, \textit{dependencies} eksternal, prosedur \textit{deployment}, serta informasi relevan lainnya yang diperlukan untuk menjalankan sistem pada \textit{environment} tujuan.

\textbf{\textit{Installation}} merupakan tahap persiapan sebelum aktivasi sistem. \textbf{\textit{Activation}} merupakan proses eksekusi perangkat lunak pada waktu tertentu, yang dapat dilakukan melalui antarmuka grafis (\textit{graphical user interface}) atau sebagai layanan latar belakang (\textit{daemon process}).

\textbf{\textit{Updating}} adalah proses penggantian komponen perangkat lunak yang terinstal dengan versi yang lebih baru. Adapun \textbf{\textit{Undeployment}} atau yang dikenal juga sebagai \textit{deinstallation}, merupakan proses penghapusan perangkat lunak yang terinstal dari suatu sistem.
\par
Menurut penelitian yang dilakukan oleh Mockus dkk. \cite{Mockus2005}, kualitas \textit{deployment} suatu perangkat lunak merupakan salah satu faktor utama yang mempengaruhi persepsi konsumen terhadap kualitas keseluruhan perangkat lunak tersebut. Hal senada diungkapkan oleh Jansen dan Brinkkemper \cite{Jansen2006} yang menyatakan bahwa kelancaran proses \textit{deployment} merupakan aspek esensial dalam meningkatkan kualitas produk perangkat lunak suatu perusahaan atau organisasi.

Namun demikian, terdapat beberapa tantangan yang sering dihadapi dalam pelaksanaan aktivitas \textit{deployment}. Menurut Carzaniga \cite{Carzaniga1998}, tantangan-tantangan tersebut meliputi:

\begin{itemize}
    \item \textbf{Penggantian Komponen}: Kesulitan dalam melakukan pembaruan (\textit{update}) terhadap komponen sistem yang sedang berjalan tanpa mengganggu layanan.

    \item \textit{\textbf{Dependencies}}: Kompleksitas ketergantungan (\textit{dependencies}) antarkomponen yang harus dikelola dengan hati-hati.

    \item \textbf{Koordinasi}: Perlunya koordinasi yang baik untuk memastikan pembaruan tidak mengganggu proses bisnis yang sedang berjalan.

    \item \textbf{Pluralitas Platform}: Tantangan dalam mengelola \textit{deployment} pada berbagai platform yang berbeda, termasuk kompatibilitas dengan berbagai sistem operasi.
\end{itemize}
\par
Seiring dengan perkembangan teknologi, kompleksitas suatu perangkat lunak cenderung mengalami peningkatan yang signifikan \cite{Tania2014, Newman2015}. Pertumbuhan tim pengembang yang diikuti dengan peningkatan kompleksitas perangkat lunak serta tantangan dalam pemeliharaan (\textit{maintainability}) seringkali menciptakan hambatan (\textit{bottleneck}) dalam siklus pengembangan. Hal ini pada akhirnya dapat menurunkan efisiensi pengembangan perangkat lunak secara keseluruhan \cite{Yale2016}.

Untuk mengatasi tantangan ini, diperlukan pendekatan arsitektur yang lebih baik, salah satunya adalah dengan menerapkan arsitektur \textit{microservice} \cite{Tania2014}. Konsep \textit{microservice} memungkinkan pengembangan perangkat lunak dengan memecahnya menjadi komponen-komponen kecil yang terpisah (\textit{loosely coupled}). Setiap komponen dapat dikembangkan, dijalankan, dan diatur secara independen oleh tim pengembang yang berbeda \cite{Xiao2017}.

Saat ini, arsitektur \textit{microservice} telah menjadi pilihan utama dalam pengembangan perangkat lunak yang membutuhkan skalabilitas tinggi \cite{Wu2014}. Awalnya, implementasi \textit{microservice} dilakukan dengan memanfaatkan beberapa \textit{Virtual Machine} (VM) yang saling berkomunikasi melalui protokol REST/HTTP (\textit{Hypertext Transfer Protocol}) atau RPC (\textit{Remote Procedure Call}) \cite{Khazaei2016}. Namun, pendekatan berbasis VM ini memiliki beberapa kelemahan, seperti kebutuhan akan operasi manual yang intensif dan biaya operasional yang relatif tinggi. Hal ini menyebabkan proses pengembangan dan penskalaan (\textit{scaling}) menjadi lebih kompleks dan memakan waktu \cite{Khazaei2016}.
\par
Perkembangan teknologi telah menggeser paradigma dari penggunaan \textit{Virtual Machine} (VM) menuju pendekatan yang lebih modern dalam merancang arsitektur \textit{microservice}. Teknologi \textit{containerization} hadir sebagai solusi yang memungkinkan pengemasan perangkat lunak dalam wadah yang ringkas dan efisien \cite{Khazaei2016}.
\par
Abstraksi yang diberikan oleh \textit{container} tidak hanya menyederhanakan proses eksekusi, tetapi juga memberikan fleksibilitas yang lebih besar dalam manajemen sumber daya. Salah satu keunggulan utamanya adalah kemudahan dalam melakukan penskalaan (\textit{scaling}), di mana penyesuaian kapasitas dapat dilakukan dengan hanya mengatur jumlah \textit{container} yang berjalan \cite{Singh2017}.
\par
\textit{Containerization} \cite{davidbritch} membawa dampak signifikan terhadap aspek infrastruktur dan \textit{runtime} dalam pengembangan perangkat lunak. Namun, kehadiran \textit{container} menuntut adanya sistem orkestrasi yang dapat mengelola \textit{container} tersebut secara otomatis.

Kubernetes hadir sebagai solusi dengan menyediakan berbagai fitur canggih, termasuk \textit{automatic scaling} yang memungkinkan penyesuaian sumber daya secara dinamis berdasarkan beban kerja \cite{Leila2018, kubernetes_2021}.

Saat ini, proses \textit{deployment} \textit{container} atau \textit{service} masih sering
kali dilakukan secara manual melalui modifikasi konfigurasi \textit{deployment} berupa file text.
Kondisi ini menimbulkan kebutuhan akan solusi otomatisasi yang dapat menangani proses
\textit{deployment} untuk \textit{service} baru atau yang mengalami perubahan (\textit{update}) secara lebih efisien.
\par
Dengan banyaknya persaingan dalam dunia perangkat lunak yang terjadi dalam waktu ini sebuah perusahaan
atau pengembang memerlukan waktu yang cepat untuk melakukan deployment sebuah perangkat lunak. Terdapat beberapa macam
workflow sebuah deploymen perangkat lunak saat ini, tetapi yang industri saat ini lakukan adalah metode DevOps \cite{Bass2018}.
Metode DevOps sendiri merupakan metode yang digunakan untuk mengembangkan perangkat lunak yang
menjembatani antara dua team yang terisolasi dalam struktur organisasi, contohnya adalah team pengembang (Dev)
dan team operasi (Ops) \cite{Bolscher2019}. Konsep DevOps \cite{Bass2018} sendiri memungkinkan team pengembang dan operasi
untuk membangun sebuah perangkat lunak yang dapat dijalankan secara otomatis dan secara berkala dengan menggunakan alat bantu DevOps.
Tujuan utama dari itu adalah untuk meningkatkan kecepatan, reliabilitas, dan perangkat lunak yang lebih baik.
Dalam DevOps sendiri terdapat beberapa sub-metode yang digunakan yaitu, \textit{Continuous Integration} (CI), \textit{Continuous DElivery} (CDE),
dan \textit{Continuous Deployment} (CD)  \cite{phoenix2013}.
\par
Meskipun DevOps menawarkan berbagai keunggulan, implementasi praktik
\textit{Continuous Integration} (CI), \textit{Continuous Delivery} (CDE), dan \textit{Continuous Deployment} (CD) menghadapi beberapa tantangan signifikan.
Tantangan-tantangan tersebut meliputi kompleksitas dalam pemilihan dan adopsi
alat bantu (\textit{tools}), kebutuhan adaptasi sumber daya manusia, serta
kerentanan terhadap kesalahan konfigurasi selama proses migrasi.

Berdasarkan tinjauan literatur,
terdapat beberapa isu krusial dalam implementasi DevOps,
antara lain: tantangan arsitektural \cite{Bolscher2019}, kompleksitas manajemen
\textit{tools} \cite{Proulx2018}, adopsi metodologi baru \cite{Abbass2019, Leite2019},
aspek keamanan dalam alur CI/CD \cite{Shahin2017}, serta mekanisme \textit{rollback}
yang efektif \cite{Fritzsch2019}. Penelitian yang dilakukan oleh Ramadoni dkk.
\cite{Ramadoni2021} mengusulkan pendekatan GitOps sebagai solusi untuk mengatasi
permasalahan mekanisme \textit{rollback} dan keamanan alur CI/CD.

Namun demikian, penelitian Ramadoni dkk. \cite{Ramadoni2021} yang menggunakan pendekatan
\textit{pull-based} belum memberikan penjelasan mendalam mengenai pertimbangan
pemilihan metode tersebut dibandingkan dengan pendekatan \textit{push-based}.
Selain itu, penelitian tersebut juga belum melakukan analisis komparatif terhadap
\textit{tools} operator GitOps, khususnya ArgoCD, dalam konteks lingkungan Kubernetes.
Oleh karena itu, penelitian ini bertujuan untuk menganalisis efektivitas berbagai
\textit{tools} operator GitOps dan membandingkan kinerja antara pendekatan
\textit{pull-based} dan \textit{push-based} dalam kerangka kerja DevOps.
% \section{Identifikasi Masalah}
% Berdasarkan latar belakang di atas dapat diidentifikasi masalah-masalah sebagai berikut:
% \begin{enumerate}[nolistsep,leftmargin=0.5cm]
%     \item Masalah 1.
%     \item Masalah 2.
%     \item Masalah 3.
% \end{enumerate}
\vspace{0.5cm}
\section{Rumusan Masalah}
Berdasarkan latar belakang masalah diatas, adapun rumusan masalah pada penelitian ini antara lain:
\begin{enumerate}[label=\alph*.]
    \item Bagaimana cara sebuah service yang source nya diubah dilakukan  deployment secara otomatis pada sistem kubernetes dengan metode GitOps ?
    \item Bagaimana cara provisioning infrastruktur yang dibutuhkan oleh service  secara otomatis pada sistem microservice yang berjalan pada kubernetes?
    \item Bagaimana perbedaan mendasar dari tools yang digunakan sebagai operator GitOps pada sistem kubernetes?
    \item Bagaimana hasil analisis tipe deployment Pull-based dan Push-based pada continous deployment?
\end{enumerate}

\vspace{0.5cm}
\section{Tujuan Penelitian}
Berdasarkan rumusan masalah yang telah diformulasikan, maka terdapat beberapa tujuan pada studi ini:
\begin{enumerate}[label=\alph*.]
    \item Melakukan implementasi continous deployment pada sebuah sistem microservice yang berjalan pada kubernetes.
    \item Melakukan analisis terhadap metode deployment secara automatis menggunakan Pull-based dan Push-based pada continuous deployment.
    \item Melanjutkan saran dari penelitian sebelumnya \cite{Ramadoni2021} dengan merancang  workflow continuous integration yang diintegrasikan dengan ArgoCD.
\end{enumerate}

% \section{Manfaat Penelitian}
% \begin{enumerate}[nolistsep,leftmargin=0.5cm]
%     \item Manfaat bagi pribadi.
%     \item Manfaat bagi institusi.
%     \item Manfaat bagi masyarakan umum.
% \end{enumerate}
\vspace{0.5cm}
\section{Batasan Masalah}
Dalam penelitian ini, peneliti akan membatasi masalah yang akan diteliti antara lain:
\begin{enumerate}[label=\alph*.]
    \item Menggunakan kubernetes sebagai alat orkestrasi \textit{container}.
    \item Implementasi CI/CD dilakukan menggunakan GitLabCI.
    \item Container runtime yang digunakan adalah Docker.
    \item perangkat lunak microservice yang digunakan berupa perangkat lunak berbasis web.
\end{enumerate}
\newpage
