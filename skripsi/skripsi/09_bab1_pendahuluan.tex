%-----------------------------------------------------------------------------%
\chapter{PENDAHULUAN}
%-----------------------------------------------------------------------------%

\vspace{4.5pt}
\setlength{\parskip}{0.5em}

\section{Latar Belakang} \label{sec:latar_belakang}
Sebuah perangkat lunak sebelum digunakan oleh pengguna secara luas perlu dilakukan proses deployment
pada perangkat lunak tersebut. Deployment sebuah perangkat lunak dapat didefinisikan sebagai akusisi
dan eksekusi sebuah perangkat lunak. Proses ini biasanya dilakukan oleh seorang software deployer atau dalam bahasa yang sering
digunakan belakangan ini seorang SRE (site reliability engineer)  \cite{Lyu2007}.
Maka dari itu dapat dikatan deployment adalah aktivitas post-production sebuah perangkat lunak untuk digunakan oleh konsumen.
Proses deployment sebuah perangkat lunak terdiri dari beberapa proses yang saling berhubungan seperti proses release sebuah perangkat lunak,
instalasi perangkat lunak kedalam environment exectuion, dan aktivasi sebuah software  \cite{Carzaniga1998}.

Untuk mendeploy sebuah sistem perangkat lunak juga ada beberapa yang harus diperhatikan antara lain sub-komponen yang dibutuhkan atau package external,
resource (hardware). Untuk melakukan deploymen sub-komponen ini dibutuhkannya sebuah konfigurasi yang mendeskripsikan versi sub-komponen yang digunakan oleh perangkat lunak.
Dengan bahasa modern sekarang konfigurasi sub-komponen yang digunakan oleh main aplikasi biasanya sudah automatis terbuat contohnya antara lain adalah;
go modules  \cite{go_mod} dalam bahasa Golang, requirement file  \cite{requirementPython} dalam bahasa Python, atau file RubySpec  \cite{rubySpec} pada bahasa Ruby.
\par
Terdapat beberapa karakteristik yang mendasar pada deployment sebuah perangkat lunak yang ditulis oleh Alan Dearle yaitu  \cite{Dearle2007};
\textbf{\textit{Release}} berupa jembatan antara proses deployment dengan proses development. yang meliputi semua operasi
yang diperlukan untuk mempersiapkan sebuah sistem  untuk di-transfer ke konsumen.
Aktivitas release ini juga menentukan resource yang dibutuhkan oleh sebuah sistem perangkat lunak untuk dapat beroperasi pada environment nya.
Setelah itu dilakukan packaging pada sistem perangkat lunak.
Package tersebut harus mengandung komponen yang dibutuhkan oleh sistem, deskripsi sistem, dependencies pada komponen eksternal, prosedur deployment,
dan semua informasi yang relevan dari sistem tersebut pada environment yang akan dijalankan.
\textbf{\textit{Installation}} diperlukan untuk persiapan melakukan activation. \textbf{\textit{Activation}} adalah proses ekseksui sebuah perangkat lunak pada waktu tertentu, biasa menggunakan grafik antar muka ataupun proses daemon.
Updating adalah proses untuk mengganti bagian dari perangkat lunak yang terinstal dengan versi yang lebih baru.
Selanjutnya yaitu \textbf{\textit{Undeployment}} yaitu proses menghapus software yang terinstall dalam sebuah mesin ini dapat disebut dengan \textit{deinstallation}.
\par
Menurut Mockus dkk  \cite{Mockus2005} kualitas deployment sebuah perangkat lunak masuk kedalam faktor utama dalam persepsi konsumer dalam hal kualitas sebuah perangkat lunak.
Jansen dan Brinkkemper  \cite{Jansen2006} juga mengatakan bahwa kelancaran sebuah deployment perangkat lunak adalah hal esensial untuk meningkatkan produk perangkat lunak sebuah perusahaan/organisasi.
Tetapi terdapat beberapa tantangan yang dihadapi pada saat melakukan aktivitas deploymet.
Menurut Antonio Carzaniga  \cite{Carzaniga1998} terdapat beberapa tantangan yang sering dihadapi pada saat melakukan deployment yaitu;
\textbf{Mengganti} atau melakukan update sebuah sistem terhadap komponen yang sudah berjalan,
\textbf{Dependencies} komponen antar satu sama lain, dan
\textbf{Koordinasi} ketika melakukan update apakah akan mengganggu proses bisnis yang sedang berjalan atau tidak, dan juga
\textbf{Mengatur} platform yang heterogen misalnya terhadap spesifik sistem operasi yang digunakan.
\par
Seiring berjalannya waktu, sebuah aplikasi cenderung menjadi semakin  kompleks  \cite{Tania2014, Newman2015}.
Dengan team pengembang dan aplikasi yang selalu bertumbuh  dari sisi kompleksitas dan \textit{maintainability} biasanya
menyebabkan model  pengembangan aplikasi menjadi susah untuk dikembangkan atau dapat dikatakan  terdapat \textit{bottleneck}
sehingga aplikasi menjadi tidak efisien  \cite{Yale2016}. Dengan berkembangnya kompleksitas sebuah aplikasi diperlukan
sebuah arsitektur yang  bisa menyelesaikan hal itu, salah satu caranya yaitu menggunakan arsitektur \textit{microservice}  \cite{Tania2014}.
Dengan \textit{microservice} aplikasi dibagi menjadi bagian-bagian kecil (unit) yang terpisah satu dengan lainnya  \cite{Tania2014}.
Masing-masing bagian aplikasi ini  dapat dijalankan dan dikembangkan secara independent (dari sisi developer)  \cite{Xiao2017}.
\par
Saat ini aplikasi berbasis \textit{microservice} menjadi pilihan sebagai arsitektur utama ketika
membangun aplikasi yang scalable  \cite{Wu2014}. Aplikasi berbasis \textit{microservice} ini
dulunya dikembangkan dengan dengan menjalankan beberapa VM  (virtual machine)
yang saling berkomunikasi satu sama lain melalui REST/HTTP  (Hypertext Transfer Protocol)
ataupun RPC \textit{(Remote Procedure Call)}  \cite{Khazaei2016}. Karena  pada saat itu deployment \textit{microservice} itu sendiri
masih berbasis VM yang  membutuhkan operasi manual dan juga biaya yang mahal, aplikasi berbasis
\textit{microservice} pun menjadi susah dan kompleks dari sisi biaya dan waktu yang  dibutuhkan untuk dilakukan
pengembangan dan juga \textit{scaling}  \cite{Khazaei2016} .
\par
\newpage
Teknologi VM sendiri sudah perlahan ditinggalkan ketika merancang \textit{microservice}.
Dengan adanya teknologi \textit{containerization} aplikasi dapat dibungkus agar dapat dijalankan dengan mudah dan efisien  \cite{Khazaei2016}.
Menjalankan aplikasi dengan  abstraksi yang diberikan oleh sebuah \textit{container} juga membawa fleksibilitas
dalam  pengelolaannya. Salah satu manfaatnya adalah scaling aplikasi yang jauh lebih  mudah, yaitu hanya
dengan melakukan penyesuaian jumlah \textit{container} yang  dijalankan \cite{Singh2017}.
\par
\textit{Containerization} \cite{davidbritch} ini sangat berdampak pada aspek infrastruktur dan \textit{runtime} sebuah aplikasi.
Tapi dengan adanya \textit{container} diperlukan juga sebuah  sistem yang melakukan orkestrasi secara otomatis pada \textit{container} tersebut.
Dengan  adanya kubernetes kita dapat memanfaatkan fitur fitur yang diberikan oleh kubernetes antara lain fitur automatic scaling \cite{Leila2018, kubernetes_2021}.
Saat ini untuk melakukan  deployment sebuah \textit{container} (\textit{service)}) dilakukan secara manual dengan merubah file deployment.
Dengan demikian, diperlukan cara otomatis untuk melakukan deployment \textit{service} yang baru/diubah.
\par
Dengan banyaknya persaingan dalam dunia perangkat lunak yang terjadi dalam waktu ini sebuah perusahaan
atau developer memerlukan waktu yang cepat untuk melakukan deployment sebuah perangkat lunak. Terdapat beberapa macam
workflow sebuah deploymen perangkat lunak saat ini, tetapi yang industri saat ini lakukan adalah metode DevOps \cite{Bass2018}.
Metode DevOps sendiri merupakan metode yang digunakan untuk mengembangkan perangkat lunak yang
menjembatani antara dua team yang terisolasi dalam struktur organisasi, contohnya adalah team pengembang (Dev)
dan team operasi (Ops) \cite{Bolscher2019}. Konsep DevOps \cite{Bass2018} sendiri memungkinkan team devloper dan operasi
untuk membangun sebuah perangkat lunak yang dapat dijalankan secara otomatis dan secara berkala dengan menggunakan alat bantu DevOps.
Tujuan utama dari itu adalah untuk meningkatkan kecepatan, reliabilitas, dan perangkat lunak yang lebih baik.
Dalam DevOps sendiri terdapat beberapa sub-metode yang digunakan yaitu, \textit{Continuous Integration} (CI), \textit{Continuous DElivery} (CDE),
dan \textit{Continuous Deployment} (CD)  \cite{phoenix2013}.
\par
Walaupun DevOps sendiri memiliki beberapa keunggulan, tetapi implementasi CI,CDE, dan CD bukanlah hal yang mudah.
Pemilihan tools dan adaptasinya, adaptasi karyawan, dan miskonfigurasi yang biasa terjadi pada saat migrasi menggunakan metode DevOps.
\newpage
Terdapat beberapa masalah yang sering terjadi pada saat menggunakan metode DevOps yaitu; Arsitektur \cite{Bolscher2019},
tools \cite{Proulx2018}, metode baru  \cite{Abbass2019, Leite2019}, Keamanan dari flow CI/CD \cite{Shahin2017}, dan mekanisme rollback \cite{Fritzsch2019}.
Pada penelitian yang dilakukan oleh Ramadoni dkk. \cite{Ramadoni2021} dilakukan analisis menggunakan metode GitOps
yang dapat menyelesaikan permasalahan mekanisme rollback, dan keamanan flow CI/CD
\par
Pada penelitian yang dilakukan oleh Ramadoni dkk. \cite{Ramadoni2021} digunakan metode pull-based untuk menyelesaikan permasalahan diatas. Pada penelitian tersebut tidak
dijelaskan perbedaan kenapa harus menggunakan metode pull-based atau push-based dan juga tidak dijelaskan
perbandingan tools yang digunakan sebagai operator GitOps yaitu ArgoCD pada sebuah cluster Kubernetes.
Maka tujuan dari penelitian ini adalah untuk melakukan analisis pada tools yang digunakan
sebagai operator GitOps dan melakukan perbandingan metode pull-based dan push-based pada metode DevOps.

% \section{Identifikasi Masalah}
% Berdasarkan latar belakang di atas dapat diidentifikasi masalah-masalah sebagai berikut:
% \begin{enumerate}[nolistsep,leftmargin=0.5cm]
%     \item Masalah 1.
%     \item Masalah 2.
%     \item Masalah 3.
% \end{enumerate}
\vspace{0.5cm}
\section{Rumusan Masalah}
Berdasarkan latar belakang masalah diatas, adapun rumusan masalah pada penelitian ini antara lain:
\begin{enumerate}[label=\alph*.]
    \item Bagaimana cara sebuah service yang source nya diubah dilakukan  deployment secara otomatis pada sistem kubernetes dengan metode GitOps ?
    \item Bagaimana cara provisioning infrastruktur yang dibutuhkan oleh service  secara otomatis pada sistem microservice yang berjalan pada kubernetes?
    \item Bagaimana perbedaan mendasar dari tools yang digunakan sebagai operator GitOps pada sistem kubernetes?
    \item Bagaimana hasil analisis tipe deployment Pull-based dan Push-based pada continous deployment?
\end{enumerate}

\vspace{0.5cm}
\section{Tujuan Penelitian}
Berdasarkan rumusan masalah yang telah diformulasikan, maka terdapat beberapa tujuan pada studi ini:
\begin{enumerate}[label=\alph*.]
    \item Melakukan implementasi continous deployment pada sebuah sistem microservice yang berjalan pada kubernetes.
    \item Melakukan analisis terhadap metode deployment secara automatis menggunakan Pull-based dan Push-based pada continuous deployment.
    \item Melanjutkan saran dari penelitian sebelumnya \cite{Ramadoni2021} dengan merancang  workflow continuous integration yang diintegrasikan dengan ArgoCD.
\end{enumerate}

% \section{Manfaat Penelitian}
% \begin{enumerate}[nolistsep,leftmargin=0.5cm]
%     \item Manfaat bagi pribadi.
%     \item Manfaat bagi institusi.
%     \item Manfaat bagi masyarakan umum.
% \end{enumerate}
\vspace{0.5cm}
\section{Batasan Masalah}
Dalam penelitian ini, peneliti akan membatasi masalah yang akan diteliti antara lain:
\begin{enumerate}[label=\alph*.]
    \item Menggunakan kubernetes sebagai alat orkestrasi \textit{container}.
    \item Implementasi CI/CD dilakukan menggunakan GitLabCI.
    \item Container runtime yang digunakan adalah Docker.
    \item Aplikasi microservice yang digunakan berupa aplikasi berbasis web.
\end{enumerate}
\newpage